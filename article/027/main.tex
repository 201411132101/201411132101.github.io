\documentclass[a4paper,11pt]{article}
\usepackage{graphicx}
\usepackage{subfigure}
\usepackage{geometry}
\usepackage{framed}
\usepackage{lipsum}
\usepackage{color}
\usepackage{CJKutf8}
\usepackage{amsmath}
\usepackage{amssymb}
\usepackage{amsthm}
\usepackage{multirow}
\usepackage{titlesec}
\usepackage{enumerate}
\usepackage{mathrsfs}


% 定义各种环境
\newtheorem{theorem}{Theorem}[section]
\newtheorem{lemma}[theorem]{Lemma}
\newtheorem{corollary}[theorem]{Corollary}
\newtheorem{proposition}[theorem]{Proposition}
\newtheorem{example}[theorem]{Example}
\theoremstyle{definition}
\newtheorem{remark}[theorem]{Remark}
\newtheorem{definition}[theorem]{Definition}
\newtheorem{assumption}[theorem]{Assumption}

\def \supp {\mathop\mathrm{\,supp\,}}

% 设置 proof 的格式
% \renewcommand{\proofname}{\emph{Proof}}

% 基本信息
\title{Bump Function, Test Function, $C_c^\infty(\mathbb{R}^n)$}

\begin{document}
\begin{CJK*}{UTF8}{gbsn}

\maketitle

Bump function, 或者说 test function, 是指 $ \mathbb{R}^n $ 上具有紧支集的光滑函数(无穷阶可微). 
将全体 bump function 记为 $ C_c^\infty(\mathbb{R}^n) $.

注: 应该有推广的情况, 暂时不关心这个.

\begin{example} \label{ex1}
    定义 $ \phi : \mathbb{R}^n \to \mathbb{R}^n $,
    $$
        \phi(x) := \left\{
        \begin{aligned}
            &\exp \left( \frac{1}{1-|x|^2} \right), && |x| < 1, \\
            &0, & & \text{otherwise}.
        \end{aligned} \right. 
    $$
    $ \phi $ 是 bump function.
\end{example}

Example \ref{ex1} 中 $ \phi $ 是 bump function 的证明有点麻烦, 暂且略去. 
下面给出赫赫有名的 Urysohn 引理.

\begin{theorem}
    设 $ F \subset \mathbb{R}^n $ 是紧集, $ G \subset \mathbb{R}^n $ 是开集且 $ F \subset G $.
    存在 $ f \in C_c^\infty(\mathbb{R}^n) $, 使得
    $ f $ 的值域为 $ [0, 1] $ 且 $ f $ 在 $ F $ 中恒等于 $ 1 $, 在 $ G^c $ 中恒等于 $ 0 $.
\end{theorem}

\begin{proof}
    设 $ \delta := d(F, G^c) $, 不难证明 $ \delta > 0 $. 
    定义 $ U := \{x \in \mathbb{R}^n : d(x, F) < \frac{\delta}{2} \} $.
    用 Example \ref{ex1} 中的 $ \phi $ 来构造卷积核 $ \varphi $, 
    对 $ \forall x \in \mathbb{R}^n $, $ \forall \varepsilon \in (0, \infty) $, 
    $$ 
        \varphi(x) := \frac{1}{\varepsilon^n} \phi \left(\frac{x}{\varepsilon}\right) / \|\phi\|_{L^1(\mathbb{R}^n)}.
    $$
    则 $ \varphi \in C_c^\infty(\mathbb{R}^n) $ 是非负函数, $ \|\varphi\|_{L^1(\mathbb{R}^n)} = 1 $.
    取 $ \varepsilon \in (0, \frac{\delta}{2}) $, 
    此时对 $ \forall x \in \mathbb{R}^n $ 且 $ |x| \geq \frac{\delta}{2} $,
    $ \varphi(x) = 0 $.
    
    取 $ f := \mathrm{1}_{U} * \varphi $, 则 $ \varphi \in C_c^\infty(\mathbb{R}^n) $.
    注意到对 $ \forall x \in \mathbb{R}^n $,
    $$
        f(x) = (\mathrm{1}_{U} * \varphi) (x) 
             =  \int_{\mathbb{R}^n} \mathrm{1}_{U}(x - y) \varphi(y) dy.
             =  \int_{|y| < \frac{\delta}{2}} \mathrm{1}_{U}(x - y) \varphi(y) dy.
    $$
    因此 $ f $ 的值域为 $ [0, 1] $ 且 $ f $ 在 $ F $ 中恒等于 $ 1 $, 在 $ G^c $ 中恒等于 $ 0 $.
\end{proof}

\end{CJK*}
\end{document} 


