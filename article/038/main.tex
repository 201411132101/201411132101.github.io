\documentclass[a4paper,11pt]{article}
\usepackage{graphicx}
\usepackage{subfigure}
\usepackage{geometry}
\usepackage{framed}
\usepackage{lipsum}
\usepackage{color}
\usepackage{amsmath}
\usepackage{amssymb}
\usepackage{amsthm}
\usepackage{multirow}
\usepackage{titlesec}
\usepackage{enumerate}
\usepackage{mathrsfs}
\usepackage{ctex}


% 定义各种环境
\newtheorem{theorem}{Theorem}[section]
\newtheorem{lemma}[theorem]{Lemma}
\newtheorem{corollary}[theorem]{Corollary}
\newtheorem{proposition}[theorem]{Proposition}
\newtheorem{example}[theorem]{Example}
\theoremstyle{definition}
\newtheorem{remark}[theorem]{Remark}
\newtheorem{definition}[theorem]{Definition}
\newtheorem{assumption}[theorem]{Assumption}

\def \supp {\mathop\mathrm{\,supp\,}}

% 设置 proof 的格式
% \renewcommand{\proofname}{\emph{Proof}}

% 基本信息
\title{投影算子}

\begin{document}


\maketitle

\section{投影算子}

\begin{definition}
    设 $ \mathcal{X} $ 是 Banach 空间. 线性映射 $ P :\ \mathcal{X} \to \mathcal{X} $ 若满足
    $$
        P^2 = P,
    $$
    则称 $ P $ 是 $ \mathcal{X} $ 上的投影算子. 记 
    $$
        \mathcal{R}(P) := \{ P x :\ x \in \mathcal{X}\} 
        \quad \text{且} \quad
        \mathcal{N}(P) := \{x \in \mathcal{X} :\ Px = \theta \}.
    $$
\end{definition}

\begin{definition}
    设 $ V_1,\ V_2 $ 是线性空间 $ V $ 的两个线性子空间, 若对 $ \forall x \in V_1 + V_2 $,
    分解 
    $$  
        x = x_1 + x_2, \quad \text{其中 } x_1 \in V_1,\ x_2 \in V_2
    $$
    是唯一的, 则称 $ V_1 + V_2 $ 是 $ V_1 $ 和 $ V_2 $ 的直和, 记为 $ V_1 \oplus V_2 $.
\end{definition}

\begin{proposition}
    设 $ \mathcal{X} $ 是 Banach 空间且 $ P $ 是 $ \mathcal{X} $ 上的投影算子. 则
    \begin{enumerate}[{\rm(i)}]
        \item $ P $ 的不动点全体就是 $ \mathcal{R}(P) $, 即对 $ \forall x \in \mathcal{R}(P) $, $ Px = x $;
        \item $ \mathcal{R}(P) = \mathcal{N}(I - P) $;
        \item $ \mathcal{R}(P) $ 和 $ \mathcal{N}(P) $ 是 $ \mathcal{X} $ 的两个线性子空间;
        \item $ \mathcal{X} = \mathcal{R}(P) \oplus \mathcal{N}(P) $.
    \end{enumerate}
\end{proposition}


\begin{theorem} \label{3}
    设 $ \mathcal{X} $ 是 Banach 空间且 $ P $ 是 $ \mathcal{X} $ 上的投影算子. 
    $ P $ 有界当且仅当 $ \mathcal{R}(P) $ 和 $ \mathcal{N}(P) $ 闭.
\end{theorem}

\begin{proof}
    先证 $ "\Rightarrow" $. $ P $ 有界, 则 $ \mathcal{N}(P) $ 闭.
    而 $ \mathcal{R}(P) = \mathcal{N}(I-P) $ 也闭, 因为 $ I-P $ 有界.
    
    再证 $ "\Leftarrow" $. 
    断言 $ P $ 是闭算子. 事实上, 设 $ \{x_n\} \subset \mathcal{X} $ 满足 $ x_n \to x_0 \text{ in } \mathcal{X} $ 且 
    $ P x_n \to y_0 \text{ in } \mathcal{X} $.
    则 $ (I - P) x_n \to x_0 - y_0 \text{ in } \mathcal{X} $.
    注意到 $ \{(I - P) x_n\}_{n \in \mathbb{N}} \subset \mathcal{N}(P) $ 且 $ \mathcal{N}(P) $ 闭,
    因此 $ x_0 - y_0 \in \mathcal{N}(P) $. 
    又因为 $ \{P x_n\}_{n \in \mathbb{N}} \subset \mathcal{R}(P) $ 且 $ \mathcal{R}(P) $ 闭,
    从而 $ y_0 \in \mathcal{R}(P) $. 综上有
    $$
        y_0 \overset{y \in \mathcal{R}(P)}{=} Py_0
          \overset{x_0 - y_0 \in \mathcal{N}(P)}{=} Py_0 + P(x_0 - y_0)
          = Px_0.
    $$
    故 $ P $ 是闭算子. 由此及闭图像定理知, $ P \in \mathscr{L}(\mathcal{X}) $. Theorem \ref{3} 证毕.
\end{proof}

\begin{remark}
    Proposition \ref{3} 中 $ \mathcal{X} $ 的完备性是必须的.
\end{remark}

\begin{proof}
    设 $ \ell_{finite}^1 := \{ \ell^1 \text{中有限项非 0 的数列} \} $, 范数与 $ \ell^1 $ 一致, 令
    $$
        P :\ \ell_{finite}^1 \to \ell_{finite}^1,\ 
            x:= \{x_k\}_{k \in \mathbb{N}} \to \left( \sum_{k = 1}^\infty x_k, 0, 0, \ldots \right).
    $$
    则 $ P $ 是投影算子且 $ \mathcal{R}(P) $ 和 $ \mathcal{N}(P) $ 闭, 但 $ P $ 无界. 
    
    下证 $ \mathcal{N}(P) $ 闭, 其余显然.
    设 $ \{x^{(n)}\}_{n \in \mathbb{N}} \subset \mathcal{N}(P) $ 在 $ \ell_{finite}^1 $ 中收敛到 $ x^{(0)} $.
    存在 $ N \in \mathbb{N} $ 使得对 $ \forall k > N $, $ x^{(0)}_k = 0 $.
    对 $ \forall n \in \mathbb{N} $,
    \begin{align*}
        \left| \sum_{k = 1}^\infty x^{(0)}_k \right|
            = \left| \sum_{k = 1}^\infty x^{(0)}_k - \sum_{k = 1}^\infty x^{(n)}_k \right| 
            \leq \left\| x^{(n)} - x^{(0)} \right\|_{\ell_{finite}^1}
            \to 0, \quad \text{ as } n \to \infty.
    \end{align*}
    故 $ \sum_{k = 1}^\infty x^{(0)}_k = 0 $, 即 $ x^{(0)} \in \mathcal{N}(P) $.
    从而 $ \mathcal{N}(P) $ 闭.
\end{proof}

\section{Hilbert 空间上的正交补}

\begin{definition}
    设 $ \mathcal{H} $ 是 Hilbert 空间且 $ M \subset \mathcal{H} $. 称
    $$
        M^\bot := \{ x \in \mathcal{H} :\ \forall y \in M,\ (x, y) = 0 \}
    $$
    为 $ M $ 的正交补.
\end{definition}

\begin{theorem}[正交分解]
    设 $ \mathcal{H} $ 是 Hilbert 空间且 $ M $ 是 $ \mathcal{H} $ 的闭线性子空间. 则
    $$
        \mathcal{H} = M \oplus M^\bot.
    $$
\end{theorem}

\begin{proposition} \label{1}
    设 $ \mathcal{H} $ 是 Hilbert 空间且 $ M \subset \mathcal{H} $. 
    则 $ M^\bot $ 是闭线性子空间且 $ M^\bot = \overline{M}^\bot $.
\end{proposition}

\begin{proposition} \label{2}
    设 $ \mathcal{H} $ 是 Hilbert 空间且 $ M $ 是 $ \mathcal{H} $ 的子集. 则
    \begin{enumerate}[{\rm(i)}]
        \item $ M \subset M^{\bot \bot} $;
        \item 若 $ M $ 是 $ \mathcal{H} $ 的线性子空间, 则 $ \overline{M} = M^{\bot \bot} $.
    \end{enumerate}
\end{proposition}

\begin{proof}
    先证 (i). 对任意 $ x \in M $, 有 
    $ x \bot M^\bot $, 从而 $ x \in \overline{M}^{\bot \bot} $,
    故 $ M \subset \overline{M}^{\bot \bot} $. (i) 证毕.
    
    再证 (ii). 直观上来看, 由正交分解有
    $$
        \overline{M} \oplus M^\bot = \mathcal{H} = M^\bot \oplus M^{\bot \bot},
    $$
    又有 $ \overline{M} \subset M^{\bot \bot} $(这个不能少), 故 $ \overline{M} = M^{\bot \bot} $.
    
    另一个证明: 由 (i) 及 $ M^{\bot \bot} $ 闭知, $ \overline{M} \subset M^{\bot \bot} $,
    故只需证 $ M^{\bot \bot} \subset \overline{M} $. 
    对任意固定 $ x \in M^{\bot \bot} $,
    由 Proposition \ref{1} 知, $ x \bot M^\bot = \overline{M}^\bot $. 由正交分解定理,
    存在 $ x_{\overline{M}} \in \overline{M} $ 和 $ x_{\overline{M}^\bot} \in \overline{M}^\bot $ 
    使得 
    $$ 
        x = x_{\overline{M}} + x_{\overline{M}^\bot}. 
    $$
    进一步由 $ x \bot \overline{M}^\bot $ 得
    $$
        0 = \left( x, x_{\overline{M}^\bot} \right)  = \left\|  x_{\overline{M}^\bot} \right\|^2
    $$
    因此
    $$
        x = x_{\overline{M}} \in \overline{M}.
    $$
    由 $ x \in M^{\bot \bot} $ 的任意性知, $ M^{\bot \bot} \subset \overline{M} $. (ii) 证毕.
    至此, Proposition \ref{2} 证毕.
\end{proof}

\begin{remark}
    $ A \oplus B = A \oplus C $ 不一定能推出 $ B = C $, 必须加上条件 $ B \subset C $.
    反例: 设 $ \mathcal{H} := \mathbb{R}^2 $, 
    $ A := \{ (t, 0) :\ t \in \mathbb{R} \} $,
    $ B := \{ (0, t) :\ t \in \mathbb{R} \} $,
    $ C := \{ (t, t) :\ t \in \mathbb{R} \} $.
    则 $ A \oplus B = A \oplus C $ 但 $ B \neq C $.
\end{remark}

\section{Hilbert 空间上的投影算子}

\begin{proposition} \label{5}
    设 $ \mathcal{H} $ 是 Hilbert 空间, $ P $ 是 $ \mathcal{H} $ 上的有界投影算子. 
    则 $ \mathcal{R}(P) = [\mathcal{N}(P)]^\bot $ 当且仅当 $ P $ 是对称算子.
\end{proposition}

\begin{proof}
    先证 $ "\Rightarrow" $. 设 $ \mathcal{R}(P) = [\mathcal{N}(P)]^\bot $, 则
    \begin{align*}
        (P x, y) &= (Px, Py + (I - P)y) 
                 = (Px, Py)  \\
                 &= (Px + (I - P)x, Py)
                 = (x, Py).
    \end{align*}
    
    再证 $ "\Leftarrow" $. 设 $ P $ 是对称算子, 则对 $ \forall x \in \mathcal{R}(P) $, $ \forall y \in \mathcal{N}(P) $,
    存在 $ z \in \mathcal{H} $ 使得 $ x = Pz $, 因此
    $$
        (x, y) = (Pz, y) = (z, Py) = 0.
    $$
    故 $ \mathcal{R}(P) \subset [\mathcal{N}(P)]^\bot $.
    又因为由正交分解有
    $$
        \mathcal{N}(P) \oplus [\mathcal{N}(P)]^\bot = \mathcal{H} = \mathcal{N}(P) \oplus \mathcal{R}(P),
    $$
    因此 $ \mathcal{R}(P) = [\mathcal{N}(P)]^\bot $.
\end{proof}

\begin{remark}
    Proposition \ref{5} 中条件 $ P \in \mathscr{L}(\mathcal{H}) $ 应该不能去掉, 反例没想到.
\end{remark}

\begin{definition}
    有界对称投影算子被称为正交投影算子(orthogonal projection),
    有界非对称投影算子被称为斜投影算子(oblique projection).
\end{definition}

\begin{remark}
    由正交分解来定义正交投影算子更为直观.

    对 $ \forall \alpha \in \mathbb{R} $,
    定义 $ P_\alpha :\ \mathbb{R}^2 \to \mathbb{R}^2,\ (x, y) \mapsto (0, \alpha x + y) $.
    则 $ P_0 $ 是正交投影算子, 对 $ \forall \alpha \in \mathbb{R} \setminus \{0\} $, $ P_\alpha $ 是斜投影算子.
\end{remark}


\end{document} 









