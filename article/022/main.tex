\documentclass[a4paper,11pt]{article}
\usepackage{graphicx}
\usepackage{subfigure}
\usepackage{geometry}
\usepackage{framed}
\usepackage{lipsum}
\usepackage{color}
\usepackage{CJKutf8}
\usepackage{amsmath}
\usepackage{amssymb}
\usepackage{amsthm}
\usepackage{multirow}
\usepackage{titlesec}
\usepackage{enumerate}
\usepackage{mathrsfs}


% 定义各种环境
\newtheorem{theorem}{Theorem}[section]
\newtheorem{lemma}[theorem]{Lemma}
\newtheorem{corollary}[theorem]{Corollary}
\newtheorem{proposition}[theorem]{Proposition}
\newtheorem{example}[theorem]{Example}
\theoremstyle{definition}
\newtheorem{remark}[theorem]{Remark}
\newtheorem{definition}[theorem]{Definition}
\newtheorem{assumption}[theorem]{Assumption}

\def \supp {\mathop\mathrm{\,supp\,}}

% 设置 proof 的格式
% \renewcommand{\proofname}{\emph{Proof}}

% 基本信息
\title{Cartesian Product and Product Space}

\begin{document}
\begin{CJK*}{UTF8}{gbsn}

\maketitle



\section{Cartesian product}

\subsection{Definition}

以下定义来自论文\cite[p.113]{m75}.

集合 $ \mathcal{X}, \mathcal{Y} $ 的笛卡尔积, 记为 $ \mathcal{X} \times \mathcal{Y} $, 是
$$ 
    \mathcal{X} \times \mathcal{Y}
        := \{(x, y) : x \in \mathcal{X}, y \in \mathcal{Y} \}.
$$
这种定义通俗易懂但不利于推广. 
$ \mathcal{X} \times \mathcal{Y} $ 的另一种定义是
$$
    \mathcal{X} \times \mathcal{Y}
        := \{ f: \{0, 1\} \to \mathcal{X} \cup \mathcal{Y} : f(0) \in \mathcal{X}, f(1) \in \mathcal{Y} \}.
$$
这种定义能自然地推广到无穷维乘积空间.

设 $ I $ 是指标集, $ \{\mathcal{X}_i\}_{i \in I} $ 是集合列.
定义
$$
    \prod_{i \in I} \mathcal{X}_i
        := \left\{ f: I \to \bigcup_{i \in I} \mathcal{X}_i 
            : \text{ for any } i \in I, f(i) \in \mathcal{X}_i \right\}.
$$

\subsection{Example}

取值为 $ \mathbb{R} $ 的数列就是一个常见的笛卡尔积, 我们通常将其简写成 $ \{a_k\}_{k = 1}^\infty $,
但事实上它的定义是
$$
    \left\{ f: \mathbb{N} \to \mathbb{R}
        : \text{ for any } i \in \mathbb{N}, f(i) \in \mathbb{R} \right\}.
$$

以下是我自己瞎写的.
单纯的集合操作空间太小了, 我们需要尝试定义集合 $ \mathcal{X} \times \mathcal{Y} $ 上的结构.
集合 $ \mathcal{X}, \mathcal{Y} $ 上一般会有拓扑结构 $ \tau_x, \tau_y $, 集合+结构=空间,
如果我们能由 $ \tau_x, \tau_y $ 诱导出 $ \mathcal{X} \times \mathcal{Y} $ 上的拓扑 $ \tau $,
那我们就能得到乘积空间 $ (\mathcal{X} \times \mathcal{Y}, \tau) $.

\section{Product Space}

\subsection{Definition}

以下定义来自\cite[p.114]{m75}.

乘积空间在集合意义下就是笛卡尔积, 关键是拓扑结构的诱导.

设 $ I $ 是指标集, $ \{\mathcal{X}_i\}_{i \in I} $ 是一列拓扑空间.
定义
$$
    \mathcal{B}
        := \left\{ \prod_{i \in I} U_i
            : \text{ for any } i \in I, U_i \text{ is open set in } \mathcal{X}_i \right\}.
$$
为笛卡尔积 $ \prod_{i \in I} \mathcal{X}_i $ 的拓扑基. 令 $ \tau $ 为拓扑基 $ \mathcal{B} $ 诱导的拓扑, i.e.
$$
    \tau := \left\{ \bigcup_{j \in J} U_j
        : \text{ for any } j \in J, U_j \in \mathcal{B} \right\}.
$$
称 $ (\prod_{i \in I} \mathcal{X}_i, \tau) $ 为乘积空间.

\subsection{Example}

取值为 $ \mathbb{R} $ 的数列也是一个常见的乘积空间, 它上面的拓扑是
$$
    \left\{\prod_{k = 1}^\infty U_k 
        : \text{ for any } k \in \mathbb{N}, U_k \text{ is open set in } \mathbb{R} \right\}.
$$


\begin{thebibliography}{99}
    \bibitem{m75} J. R. Munkres, Topology, Second edition, 
    Prentice-Hall Inc., Englewood Cliffs, 1975.
    
\end{thebibliography}

\end{CJK*}
\end{document} 

