\documentclass[a4paper,11pt]{article}
\usepackage{graphicx}
\usepackage{subfigure}
\usepackage{geometry}
\usepackage{framed}
\usepackage{lipsum}
\usepackage{color}
\usepackage{CJKutf8}
\usepackage{amsmath}
\usepackage{amssymb}
\usepackage{amsthm}
\usepackage{multirow}
\usepackage{titlesec}
\usepackage{enumerate}
\usepackage{mathrsfs}


% 定义各种环境
\newtheorem{theorem}{Theorem}[section]
\newtheorem{lemma}[theorem]{Lemma}
\newtheorem{corollary}[theorem]{Corollary}
\newtheorem{proposition}[theorem]{Proposition}
\newtheorem{example}[theorem]{Example}
\theoremstyle{definition}
\newtheorem{remark}[theorem]{Remark}
\newtheorem{definition}[theorem]{Definition}
\newtheorem{assumption}[theorem]{Assumption}

\def \supp {\mathop\mathrm{\,supp\,}}

% 设置 proof 的格式
% \renewcommand{\proofname}{\emph{Proof}}

% 基本信息
\title{Calder\'on-Zygmund Decomposition}

\begin{document}
\begin{CJK*}{UTF8}{gbsn}

\maketitle

Calder\'on-Zygmund Decomposition 有多种类似的形式, 这里只说其中一种, 来自\cite[Theorem 2.11]{d01}.
首先交代一下符号. $ \mathcal{D} $ 是二进方体全体, $ \mathcal{D}_k $ 是第 $ k $ 层二进方体.

\begin{theorem} \label{C-Z decomposition}
    设 $ f $ 是非负可积函数, 则对任意给定的正常数 $ \lambda $, 存在两两不交的二进方体序列 $ \{Q_j\}_{j \in J} $ 
    使得
    \begin{enumerate}[{\rm(i)}]
        \item 对 a.e. $ x \notin \bigcup_{j \in J} Q_j $, $ f(x) < \lambda $;
        \item $ | \bigcup_{j \in J} Q_j | \leq \frac{1}{\lambda} \| f \|_{L^1(\mathbb{R}^n)} $;
        \item 对 $ \forall j \in J $, 
            $$
                \lambda < \frac{1}{|Q_j|} \int_{Q_j} f(x) dx < 2^n \lambda.
            $$
    \end{enumerate}
\end{theorem}

这个定理证明的关键是利用
$$
    \lambda < \frac{1}{|Q_j|} \int_{Q_j} f(x) dx,
$$
把二进方体序列构造出来, 然后证明该序列满足其它的性质.

\begin{proof}[Proof of Theorem \ref{C-Z decomposition}]
    定义 $  $
\end{proof}


\begin{thebibliography}{99}
    \bibitem{d01} J. Duoandikoetxea, Fourier analysis, American Mathematical Soc., 2001.
\end{thebibliography}

\end{CJK*}
\end{document} 

