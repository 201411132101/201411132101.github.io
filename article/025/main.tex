\documentclass[a4paper,11pt]{article}
\usepackage{graphicx}
\usepackage{subfigure}
\usepackage{geometry}
\usepackage{framed}
\usepackage{lipsum}
\usepackage{color}
\usepackage{CJKutf8}
\usepackage{amsmath}
\usepackage{amssymb}
\usepackage{amsthm}
\usepackage{multirow}
\usepackage{titlesec}
\usepackage{enumerate}
\usepackage{mathrsfs}


% 定义各种环境
\newtheorem{theorem}{Theorem}[section]
\newtheorem{lemma}[theorem]{Lemma}
\newtheorem{corollary}[theorem]{Corollary}
\newtheorem{proposition}[theorem]{Proposition}
\newtheorem{example}[theorem]{Example}
\theoremstyle{definition}
\newtheorem{remark}[theorem]{Remark}
\newtheorem{definition}[theorem]{Definition}
\newtheorem{assumption}[theorem]{Assumption}

\def \supp {\mathop\mathrm{\,supp\,}}

% 设置 proof 的格式
% \renewcommand{\proofname}{\emph{Proof}}

% 基本信息
\title{Calder\'on-Zygmund Decomposition}

\begin{document}
\begin{CJK*}{UTF8}{gbsn}

\maketitle

Calder\'on-Zygmund Decomposition 由 Calderon 和 Zygmund 在 \cite{cz52} 中引入.

\section{Calder\'on-Zygmund Lemma}

\begin{theorem}[Calder\'on-Zygmund Lemma] \label{C-Z lemma}
    设 $ f $ 是非负可积函数, 则对任意给定的正常数 $ \lambda $, 存在两两不交的二进方体序列 $ \{Q_j\}_{j \in J} $ 
    使得
    \begin{enumerate}[{\rm(i)}]
        \item 对 a.e. $ x \notin \bigcup_{j \in J} Q_j $, $ f(x) < \lambda $;
        \item $ | \bigcup_{j \in J} Q_j | \leq \frac{1}{\lambda} \| f \|_{L^1(\mathbb{R}^n)} $;
        \item 对 $ \forall j \in J $, 
            $$
                \lambda < \frac{1}{|Q_j|} \int_{Q_j} f(t) dt < 2^n \lambda.
            $$
    \end{enumerate}
\end{theorem}

\subsection{本人觉得很自然的证明}

此证明来自 \cite[Lemma 1.2]{kk13}. 
在证明 Calder\'on-Zygmund Lemma 之前, 首先回顾一下经典的 Lebesgue differentiation theorem.

\begin{theorem}[Lebesgue differentiation theorem]
    设 $ f \in L^1_{loc}(\mathbb{R}^n) $, 则对 a.e. $ x \in \mathbb{R}^n $, 
    $$
        \lim_{r \to 0^+} \frac{1}{|B(x, r)|} \int_{B(x, r)} |f(t) - f(x)| dt = 0.
    $$
\end{theorem}

\begin{remark}
    显然对 a.e. $ x \in \mathbb{R}^n $,
    $$
        \lim_{r \to 0^+} \frac{1}{|B(x, r)|} \int_{B(x, r)} f(t) dt = f(x).
    $$
    事实上, 里面的 $ B(x, r) $ 可以换成 $ B \ni x $, 球也可以换成方体.
\end{remark}

\begin{proof}[Proof of Theorem \ref{C-Z lemma}]
    设 $ \mathcal{D} $ 为二进方体全体, 考虑 $ \mathcal{D} $ 的子集
    $$
        D_{f, \lambda} := \left\{ Q \in \mathcal{D} : \frac{1}{|Q|} \int_{Q} f(t) dt > \lambda \right\}.
    $$
    称 $ Q_0 $ 是 $ D_{f, \lambda} $ 中的极大方体, 若对 $ \forall Q \in D_{f, \lambda} \setminus \{Q_0\} $, 
    $ Q_0 \not\subset Q $. 将 $ D_{f, \lambda} $ 中的全体极大方体记为 $ \{Q_j\}_{j \in J} $. 
    下证 $ \{Q_j\}_{j \in J} $ 满足 (i), (ii), (iii). 
    先证 (iii),
    对 $ \forall j \in J $, 设 $ \widetilde{Q}_j \in \mathcal{D} $ 满足 $ Q_j \subset \widetilde{Q}_j $
    且 $ |\widetilde{Q}_j| = 2^n |Q_j| $. 由此及 $ Q_j $ 是 $ D_{f, \lambda} $ 中的极大方体知
    $$
        \frac{1}{|Q_j|} \int_{Q_j} f(t) dt 
            \leq \frac{|\widetilde{Q}_j|}{|Q_j|} \frac{1}{|\widetilde{Q}_j|} \int_{|\widetilde{Q}_j|} f(t) dt 
            \leq 2^n \lambda.
    $$
    由此及 $ Q_j \in D_{f, \lambda} $ 知, (iii) 成立. 
    再证 (ii), 
    由 (iii) 知, 对 $ \forall j \in J $, 
    \begin{equation} \label{equ1}
        |Q_j| < \frac{1}{\lambda} \int_{Q_j} f(t) dt.
    \end{equation}
    断言, $ \{Q_j\}_{j \in J} $ 两两不交. 事实上, 对 $ \forall i, j \in J $ 且 $ i \neq j $, 
    $ Q_i, Q_j $ 是 $ D_{f, \lambda} $ 中的极大方体, 故 $ Q_i \not\subset Q_j $ 且 $ Q_j \not\subset Q_i $.
    由此及 $ Q_i, Q_j \in \mathcal{D} $ 知, $ Q_i \cap Q_j = \emptyset $, 断言成立. 
    由断言及 \eqref{equ1} 知,
    $$
         \left| \bigcup_{j \in J} Q_j \right| 
            =  \sum_{j \in J} |Q_j|
            < \sum_{j \in J} \frac{1}{\lambda} \int_{Q_j} f(t) dt
            =\frac{1}{\lambda} \int_{\bigcup_{j \in J} Q_j} f(t) dt
            \leq \frac{1}{\lambda} \| f \|_{L^1(\mathbb{R}^n)},
    $$
    (ii) 成立. 最后证 (i) 成立, 对 $ \forall x \notin \bigcup_{j \in J} Q_j $,
    由 $ \{Q_j\}_{j \in J} $ 的构造知, 对 $ \forall Q \in \mathcal{D} $ 且 $ Q \ni x $, 
    \begin{equation} \label{equ2}
        \frac{1}{|Q|} \int_{Q} f(t) dt \leq \lambda.
    \end{equation}
    否则 $ Q \in D_{f, \lambda}$, 从而 $ x \in \bigcup_{j \in J} Q_j $, 矛盾.
    由 Lebesgue differentiation theorem 及 \eqref{equ2} 知, 对 a.e. $ x \in \mathbb{R}^n $,
    $$
        f(x) = \lim_{Q \ni x, |Q| \to 0^+} \frac{1}{|Q|} \int_{Q} f(t) dt \leq \lambda.
    $$
    (i) 成立. 综上, Theorem \ref{C-Z lemma} 证毕.
\end{proof}

\subsection{一种挺奇怪的证明}

该证明来自\cite[Theorem 2.11]{d01}.
首先交代一下符号. $ \mathcal{D} $ 是二进方体全体, $ \mathcal{D}_k $ 是第 $ k $ 层二进方体. 
对 $ \forall \, k \in \mathbb{Z}, f \in L^1_{loc}(\mathbb{R}^n) $, 定义
$$
    E_k f (x) := \sum_{Q \in \mathcal{D}_k} \left[ \frac{1}{|Q|} \int_{Q} f(t) dt \right] \mathrm{1}_Q (x).
$$
$ E_k $ 算子是连续函数的离散化. 由此可以定义二进极大函数
$$
    M_d f(x) := \sup_{k \in \mathbb{Z}} |E_k f (x)|.
$$

\begin{theorem} \label{thm} 
    对于 $ E_k $ 和 $ M_d $, 有如下两个结论:
    \begin{enumerate}[{\rm(i)}]
        \item $ M_d $ 是弱 $ (1, 1) $ 型算子;
        \item 若 $ f \in L^1_{loc}(\mathbb{R}^n) $, 则对 a.e. $ x \in \mathbb{R}^n $,
        $$
            \lim_{k \to \infty} E_k f (x) = f(x).
        $$
    \end{enumerate}
\end{theorem}

证明的关键是利用二进极大算子, 好吧, 还是有点不懂怎么想到的.

\begin{proof}[Proof of Theorem \ref{C-Z lemma}]
    略.
\end{proof}

\section{Calder\'on-Zygmund Decomposition}

暂时没有用到, 鸽!

\begin{thebibliography}{99}

    \bibitem{cz52} A. Calderon and A. Zygmund, On the existence of certain singular integrals, 
        Acta Math. 88 (1952), 85--139.
        
    \bibitem{kk13} S. Kislyakov, N. Kruglyak,
    Extremal Problems in Interpolation Theory, Whitney-Besicovitch Coverings, and Singular Integrals,
    Birkh\"{a}user/Springer Basel AG, Basel, 2013.
    
    \bibitem{d01} J. Duoandikoetxea, Fourier analysis, American Mathematical Soc., 2001.
    
\end{thebibliography}

\end{CJK*}
\end{document} 


