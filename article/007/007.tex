\documentclass[a4paper,12pt]{article}
\usepackage{graphicx}
\usepackage{subfigure}
\usepackage{geometry}
\usepackage{framed}
\usepackage{lipsum}
\usepackage{color}
\usepackage{amsmath}
\usepackage{amssymb}
\usepackage{amsthm}
\usepackage{multirow}
\usepackage{titlesec}
\usepackage{enumerate}
\usepackage{mathrsfs}


\newtheorem{theorem}{Theorem}
\newtheorem{lemma}{Lemma}
\newtheorem{definition}{Definition}
\newtheorem{case}{Case}


% 基本信息
\title{Lebesgue's number lemma}
\author{}
\date{}

\begin{document}

\maketitle

In \cite[Lemma 7.2]{m75}, Munkres proof the following Lebesgue's number lemma.

\begin{theorem}
    If the metric space $ (X, d) $ is compact and an open cover of $ X $ is given, 
    then there exists $ \delta \in (0, \infty) $ such that every subset of $ X $ 
    having diameter less than $ \delta $ is contained in some member of the cover.
    Such a number $ \delta $ is called a Lebesgue number of this cover. 
\end{theorem}

\begin{proof}
    Let $ \mathcal{U} $ be an open cover of $ X $. Since $ X $ is compact 
    we can extract a finite subcover $ \{A_{1},\dots ,A_{n}\} \subseteq \mathcal{U} $. 
    If there exists $ k_0 \in \{1, \ldots, n\} $ such that $ A_{k_0} = X $, 
    then any $ \delta \in (0, \infty) $ will serve as a Lebesgue number. 
    Otherwise for any $ k \in \{1, \dots, n\} $, let $ C_k := X \setminus A_k $, 
    note that $ C_k $ is not empty closed set, and define a function
    $$
        f: X \rightarrow \mathbb{R}, x \mapsto \frac{1}{n} \sum_{k = 1}^n d(x, C_k).
    $$
    Since $ f $ is continuous on a compact set, let $ f $ attains a minimum $ \delta $ at $ x_0 \in X $. 
    From this and $ \{A_1, \dots, A_n\} $ is an open cover of $ X $, there exists $ k \in \{1, \ldots, n\} $ 
    such that $ x_0 \in A_k $, then $ x_0 \notin C_k $ so that
    $$
        \delta = f(x_0) \geq \frac{1}{n} d(x_0, C_k) > 0.
    $$
    Now we can verify that this $ \delta $ is the desired Lebesgue number. 
    In fact, for any $ Y \subset X $ of diameter less than $ \delta $, let $ y \in Y $, we have 
    \begin{equation} \label{equ}
        Y \subseteq B(y, \delta).
    \end{equation} 
    Since $ f(y) \geq \delta $, there exist $ k \in \{1, \ldots, n\} $ 
    such that $ d(y, C_k) \geq \delta $. It means that $ B(y, \delta) \subset X \setminus C_k $,
    then $ B(y, \delta) \subseteq A_k $. From this and \eqref{equ}, $ Y \subseteq A_k $. 
    So $ \delta $ is a Lebesgue number of cover $ \mathcal{U} $.
\end{proof}

\begin{thebibliography}{99}

    \bibitem{m75} J. R. Munkres, Topology: a first course, Prentice-Hall, Inc., Englewood Cliffs, N.J.,  1975.
    
\end{thebibliography}

\end{document} 






