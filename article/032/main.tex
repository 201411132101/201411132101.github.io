\documentclass[a4paper,11pt]{article}
\usepackage{graphicx}
\usepackage{subfigure}
\usepackage{geometry}
\usepackage{framed}
\usepackage{lipsum}
\usepackage{color}
\usepackage{CJKutf8}
\usepackage{amsmath}
\usepackage{amssymb}
\usepackage{amsthm}
\usepackage{multirow}
\usepackage{titlesec}
\usepackage{enumerate}
\usepackage{mathrsfs}


% 定义各种环境
\newtheorem{theorem}{Theorem}[section]
\newtheorem{lemma}[theorem]{Lemma}
\newtheorem{corollary}[theorem]{Corollary}
\newtheorem{proposition}[theorem]{Proposition}
\newtheorem{example}[theorem]{Example}
\theoremstyle{definition}
\newtheorem{remark}[theorem]{Remark}
\newtheorem{definition}[theorem]{Definition}
\newtheorem{assumption}[theorem]{Assumption}

\def \supp {\mathop\mathrm{\,supp\,}}

% 设置 proof 的格式
% \renewcommand{\proofname}{\emph{Proof}}

% 基本信息
\title{Borel Algebra 的平移不变性和伸缩不变性}

\begin{document}
\begin{CJK*}{UTF8}{gbsn}

\maketitle

\begin{definition}[W. Rudin, Real and Complex Analysis, p.12]
    设 $ (X, \tau) $ 是拓扑空间. 称包含 $ \tau $ 的最小 $ \sigma $-algebra 为 $ X $ 中的 Borel algebra. 
    称 Borel algebra 中的元素为 Borel set.
\end{definition}

\begin{lemma} \label{lem}
    设 $ \mathfrak{M} $ 是 $ \mathbb{R}^n $ 中的 $ \sigma $-algebra. 则
    \begin{enumerate}[{\rm (i)}]
        \item 对 $ \forall r \in \mathbb{R}^n $, $ \mathfrak{M} + r $ 是 $ \sigma $-algebra;
        \item 对 $ \forall t \in \mathbb{R} \setminus \{0\} $, $ t\mathfrak{M} $ 是 $ \sigma $-algebra.
    \end{enumerate}
\end{lemma}

\begin{proof}
    $ \mathfrak{M} $ 是 $ \mathbb{R}^n $ 中的 $ \sigma $-algebra, 因此
    
    \begin{enumerate}[{\rm (i)}]
        \item $ \mathbb{R}^n \in \mathfrak{M} $, 故 $ \mathbb{R}^n = \mathbb{R}^n + r \in \mathfrak{M} + r $;
        
        \item 对 $ \forall A \in \mathfrak{M} + r $, 有 $ A - r \in \mathfrak{M} $, 故
        $ A^c - r = (A - r)^c \in \mathfrak{M} $, 从而 $ A^c \in \mathfrak{M} + r $;
        
        \item 若 $ \{A_n\}_{n \in \mathbb{N}} \subset \mathfrak{M} + r $, 则
        $ \{A_n - r \}_{n \in \mathbb{N}} \subset \mathfrak{M} $, 故
        $$ 
            \left( \bigcup_{n \in \mathbb{N}} A_n \right) - r 
                = \bigcup_{n \in \mathbb{N}} (A_n - r) \in \mathfrak{M},
        $$
        从而 $ \cup_{n \in \mathbb{N}} A_n \in \mathfrak{M} + r $.
    \end{enumerate}
    
    综上, $ \mathfrak{M} + r $ 也是 $ \sigma $-algebra, (i) 得证. 
    (ii) 类似可证.
\end{proof}

\begin{lemma} \label{lem2}
    设 $ \mathcal{U} $ 是 $ \mathbb{R}^n $ 中全体开集构成的集合. 则
    \begin{enumerate}[{\rm (i)}]
        \item 对 $ \forall r \in \mathbb{R}^n $, $ \mathcal{U} + r = \mathcal{U} $;
        \item 对 $ \forall t \in \mathbb{R} \setminus \{0\} $, $ t \mathcal{U} = \mathcal{U} $.
    \end{enumerate}
\end{lemma}

\begin{proof}
    断言: 对 $ \forall A \in \mathcal{U} $, $ \forall r \in \mathbb{R}^n $, $ A + r \in \mathcal{U} $.
    事实上, 对 $ \forall x \in A + r $, 有 $ x - r \in A $, 因为 $ A $ 是开集, 存在 $ \delta \in (0, \infty) $
    使得 $ B(x - r, \delta) \subset A $, 从而 $ B(x, \delta) \subset A + r $, 故 $ A + r $ 是开集, 断言成立.
    由断言易证 $ \mathcal{U} + r = \mathcal{U} $, (i) 得证. 
    (ii) 类似可证.
\end{proof}

\begin{theorem} \label{thm1}
    设 $ \mathscr{B} $ 是 $ \mathbb{R}^n $ 中的 Borel algebra.
    \begin{enumerate}[{\rm (i)}]
        \item 对 $ \forall r \in \mathbb{R}^n $, $ \mathscr{B} + r = \mathscr{B} $;
        \item 对 $ \forall t \in \mathbb{R} \setminus \{0\} $, $ t \mathscr{B} = \mathscr{B} $.
    \end{enumerate}
\end{theorem}

\begin{proof}
    对 $ \forall r \in \mathbb{R}^n $, 由 Lemma \ref{lem} 和 Lemma \ref{lem2}  知, 
    $ \mathscr{B} + r $ 是 $ \sigma $-algebra 且包含 $ \mathbb{R}^n $ 中的所有开集.
    由此及 Borel algebra 定义知, $ \mathscr{B} \subset \mathscr{B} + r $.
    取 $ r := -r $ 得 $ \mathscr{B} \subset \mathscr{B} - r $,
    因此 $ \mathscr{B} + r \subset \mathscr{B} $.
    综上, $ \mathscr{B} + r = \mathscr{B} $, (i) 得证. 
    (ii) 类似可证.
\end{proof}

\begin{theorem} \label{thm2}
    一般拓扑空间的 Borel algebra 没有上述性质. 就算是 $ \mathbb{R}^n $ 也不一定.
\end{theorem}

\begin{proof}
    一般拓扑空间可能没有加法和数乘.
    
    设 $ \tau := \{ \emptyset, B(0,1), B(0,1)^c, \mathbb{R}^n \} $, 则 $ (\mathbb{R}^n, \tau) $ 是个拓扑空间.
    此时 $ \mathbb{R}^n $ 中的 Borel algebra $ \mathscr{B} = \tau $ 不满足 
    $$ 
        \mathscr{B} + r = \mathscr{B} 
            \quad \text{and} \quad 
        t \mathscr{B} = \mathscr{B}.
    $$
\end{proof}

\end{CJK*}
\end{document} 


