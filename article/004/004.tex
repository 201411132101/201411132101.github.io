\documentclass[a4paper,12pt]{article}
\usepackage{graphicx}
\usepackage{subfigure}
\usepackage{geometry}
\usepackage{framed}
\usepackage{lipsum}
\usepackage{color}
\usepackage{amsmath}
\usepackage{amssymb}
\usepackage{amsthm}
\usepackage{multirow}
\usepackage{titlesec}
\usepackage{enumerate}
\usepackage{mathrsfs}
\usepackage{fontspec}

\setmainfont{Times New Roman}

\newtheorem{theorem}{Theorem}
\newtheorem{lemma}{Lemma}
\newtheorem{definition}{Definition}
\newtheorem{case}{Case}
\newtheorem{corollary}{Corollary}


% 基本信息
\title{Equidistributed}
\author{}
\date{}

\begin{document}

\maketitle

The content is from \cite[Chapter 4, Section 2]{ss11}.

\begin{definition}
    A sequence of number $ \{ \xi_n \}_{n \in \mathbb{N}} \subset [0, 1) $ is said to be equidistributed 
    if for any interval $ (a, b) \subset [0, 1) $, 
    $$ 
        \lim_{N \to \infty} \frac{ \# \{1 \leq n \leq N: \, \xi_n \in (a, b) \}}{N} = b - a. 
    $$
where the $ \# A $ denotes the cardinality of the finite set $ A $.
\end{definition}

\begin{definition}
    Let $ [x] $ denote the greatest integer less than or equal to x and call $ [x] $ the integer part of $ x $.
    Let $ \langle x \rangle := x - [x] $ and call $ \langle x \rangle $ the fractional part of $ x $.
\end{definition}

\begin{theorem}[Weyl's criterion]
    A sequence of number $ \{ \xi_n \}_{n \in \mathbb{N}} \subset [0, 1) $ is equidistributed 
    if and only if for any $ k \in \mathbb{Z} \setminus \{0\} $,
    $$ 
        \frac{1}{N}\sum_{n=1}^N e^{2 \pi i k \xi_{n}} \to 0, \quad as \ N \to \infty. 
    $$
\end{theorem}

From Weyl's criterion, we have following corollaries.

\begin{corollary}
    Let $ P(n) := c_n x^n + \cdots + c_0 $, for any $ k \in \{0, \ldots, n\} $, $ c_n \in \mathbb{R} $
    and one of them is irrational number. Then $ \{ \langle P(n) \rangle \}_{n \in \mathbb{N}} \subset [0, 1) $ is equidistributed.
\end{corollary}

\begin{corollary}
    $ \{ \sin n \}_{n \in \mathbb{N}} $ is dense in $ [0, 1] $.
\end{corollary}

\begin{thebibliography}{99}

    \bibitem{ss11} E. M. Stein and R. Shakarchi, Fourier Analysis: An Introduction, Princeton University Press, 2011.
    
\end{thebibliography}

\end{document} 






