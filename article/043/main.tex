\documentclass[a4paper,11pt]{article}
\usepackage{graphicx}
\usepackage{subfigure}
\usepackage{geometry}
\usepackage{framed}
\usepackage{lipsum}
\usepackage{color}
\usepackage{amsmath}
\usepackage{amssymb}
\usepackage{amsthm}
\usepackage{multirow}
\usepackage{titlesec}
\usepackage{enumerate}
\usepackage{mathrsfs}
\usepackage{ctex}


\theoremstyle{definition}
\newtheorem{theorem}{Theorem}
\newtheorem{lemma}[theorem]{Lemma}
\newtheorem{corollary}[theorem]{Corollary}
\newtheorem{proposition}[theorem]{Proposition}
\newtheorem{example}[theorem]{Example}
\newtheorem{remark}[theorem]{Remark}
\newtheorem{definition}[theorem]{Definition}
\newtheorem{assumption}[theorem]{Assumption}



\def \supp {\mathop\mathrm{\,supp\,}}


% 基本信息
\title{指数迭代问题}

\begin{document}

\maketitle

\section{提出问题}

导师收到一个有意思的问题. 设 $ b \in (0, \infty) $, 则
\begin{equation} \label{1}
    b^{b^{b^{b \cdots}}} = ?
\end{equation}
这个问题乍一看很简单, 我都没仔细想, 后来师兄整出来部分结果, 
我看了看, 发现没有那么平凡, 随即研究了一番,
发现2000 年, Wassell 在 \cite{w00} 中给出了完整的回答.
然而他的证明看不懂, 故我用自己的方法证了一遍.

\begin{theorem}[Wassell] \label{2}
    设 $ b \in (0, \infty) $. 则 \eqref{1} 收敛当且仅当 $ b \in [(1/e)^e, e^{1/e}] $. 并且, 
    \begin{enumerate}[{\rm (i)}]
        \item 当 $ b \in (0, (1/e)^e) $ 时, \eqref{1} 在 $ x = b^{b^x} $ 的最大解和次大解之间循环;
        \item 当 $ b \in [(1/e)^e, 1] $ 时, \eqref{1} 收敛到 $ x = b^x $ 的唯一解;
        \item 当 $ b \in (1, e^{1/e}) $ 时, \eqref{1} 收敛到 $ x = b^x $ 的较小解;
        \item 当 $ b = e^{1/e} $ 时, \eqref{1} 收敛到 $ e $;
        \item 当 $ b \in (e^{1/e}, \infty) $ 时, \eqref{1} 趋于无穷.
    \end{enumerate}
\end{theorem}

\section{定理的证明}

令 $ a_1 := b $, 对 $ \forall k \in \mathbb{N} $, $ a_{k + 1} := b^{a_k} $.
则 \eqref{1} 收敛即 $ \{a_k\}_{k \in \mathbb{N}} $ 收敛.

注意到, 当 $ b \in (1, \infty) $ 时,
$$
    b < b^b < b^{b^b} < \cdots,
$$
即 $ \{a_k\}_{k \in \mathbb{N}} $ 单调.
这是因为由数学归纳法可得, 对 $ \forall k \in \mathbb{N} $,
$$
    a_{k + 1} = b^{a_k} < b^{a_{k + 1}} = a_{k + 2}.
$$
故下面分两种情况证明定理 \ref{2}.

\subsection{$b \in (1, \infty)$}

若 $ \{a_k\}_{k \in \mathbb{N}} $ 有界, 则单调有界必收敛, 记
$$
    x := b^{b^{b^{b \cdots}}},
$$
从而
\begin{equation} \label{3}
    x = b^x.
\end{equation}
注意到仅当 $ b \in (1, e^{1/e}] $ 时, \eqref{3} 有解, 
故当 $ b \in (e^{1/e}, \infty) $ 时, $ \{a_k\}_{k \in \mathbb{N}} $ 趋于无穷.

下证 $ b \in (1, e^{1/e}] $ 时, $ \{a_k\}_{k \in \mathbb{N}} $ 收敛.
由数学归纳法可证, 对 $ \forall k \in \mathbb{N} $,
$$
    a_{k + 1} = b^{a_k} < \left( e^{1/e} \right)^e = e.
$$
因此 $ \{a_k\}_{k \in \mathbb{N}} $ 单调有界必收敛. 

当 $ b \in (1, e^{1/e}) $ 时, \eqref{3} 有两解, 一个大于 $ e $, 一个小于 $ e $,
由此及 $ e $ 是 $ \{a_k\}_{k \in \mathbb{N}} $ 上界知, 
$ \{a_k\}_{k \in \mathbb{N}} $ 收敛到 \eqref{3} 的较小解.
当 $ b = e^{1/e} $ 时, \eqref{3} 有唯一解 $ e $, 
故 $ \{a_k\}_{k \in \mathbb{N}} $ 收敛到 $ e $.

\subsection{$b \in (0, 1]$}

证明思路源自求一个类似 $ a_1 = a $, $ a_{k + 1} = a_k + \frac{b}{a_k} $ 数列的极限,
那个数列也是震荡的, 但按奇偶单调.

由数学归纳法可得 $ \{a_{2k-1}\}_{k \in \mathbb{N}} $ 单增有上界 $ 1 $, 
$ \{a_{2k}\}_{k \in \mathbb{N}} $ 单减有下界 $ 0 $.
故可记 
$$
    x_1 := \lim_{k \in \mathbb{N}} a_{2k-1}
        \quad \text{and} \quad
    x_2 := \lim_{k \in \mathbb{N}} a_{2k}.
$$
注意到 $ x_1,\ x_2 $ 是方程
\begin{equation} \label{6}
    x = b^{b^x} 
\end{equation}
的解. 注意到 \eqref{6} 的解必属于 $ (0, 1) $, 故\eqref{6} 的解也是
\begin{equation} \label{7}
    \log (- \log x) = x \log b +  \log (- \log b)
\end{equation}
的解, 其中 $ x \in (0, 1) $.

令 $ f(x) := \log (- \log x) - x \log b - \log (- \log b) $, $ x \in (0, 1) $.

Case 1) $ b \in [(1/e)^e, 1] $. 此时, 由单调性知 \eqref{7} 有唯一解, 故
$ \{a_k\}_{k \in \mathbb{N}} $ 收敛到 \eqref{6} 的唯一解.

Case 2) $ b \in (0, (1/e)^e) $. 此时, 由单调性及 
$ f(1) > 0 $, $ f(b) < 0 $, $ f(1/e) > 0 $, $ f(b^b) < 0 $ 知 \eqref{7} 有三解, 且 
$$ 
    t_1 < b < t_2 < t_3 < b^b,
$$
其中 $ t_1,\ t_2,\ t_3 $ 是 \eqref{7} 的解.
由此及数学归纳法可证, 对 $ \forall k \in \mathbb{N} $,
$$
    a_{2k+1} = b^{b^{a_{2k-1}}} < b^{b^{t_2}} = t_2
        \quad \text{and} \quad
    a_{2k+2} = b^{b^{a_{2k}}} > b^{b^{t_3}} = t_3.
$$
因此 $ \{a_{2k-1}\}_{k \in \mathbb{N}} $ 收敛到 \eqref{7} 的次大解, 
$ \{a_{2k}\}_{k \in \mathbb{N}} $ 收敛到 \eqref{7} 的最大解.  
至此, 定理 \ref{1} 证毕.

\begin{thebibliography}{99}  
	\bibitem{w00}
    S. R. Wassell, 
    Superexponentiation and fixed points of exponential and logarithmic functions,
    Math. Mag. 73 (2000), 111--119.
\end{thebibliography}

\end{document} 









