\documentclass[a4paper,12pt]{article}
\usepackage{graphicx}
\usepackage{subfigure}
\usepackage{geometry}
\usepackage{framed}
\usepackage{lipsum}
\usepackage{color}
\usepackage{CJKutf8}
\usepackage{amsmath}
\usepackage{amssymb}
\usepackage{amsthm}
\usepackage{multirow}
\usepackage{titlesec}
\usepackage{enumerate}
\usepackage{mathrsfs}

% 页边距
\geometry{left=3.18cm,right=3.18cm,top=2.54cm,bottom=2.54cm}

% 定义各种环境
\newtheorem{theorem}{Theorem}[section]
\newtheorem{lemma}[theorem]{Lemma}
\newtheorem{corollary}[theorem]{Corollary}
\newtheorem{proposition}[theorem]{Proposition}
\newtheorem{example}[theorem]{Example}
\theoremstyle{definition}
\newtheorem{remark}[theorem]{Remark}
\newtheorem{definition}[theorem]{Definition}
\newtheorem{assumption}[theorem]{Assumption}


% 设置 proof 的格式
% \renewcommand{\proofname}{\emph{Proof}}

% 基本信息
\title{The relationships between I-AB and I-BA}

\begin{document}
\begin{CJK*}{UTF8}{gbsn}

\maketitle

\begin{theorem} \label{thm1}
    Let $ A, B \in M_n(\mathbb{C}) $.
    $$
        I_n-AB \text{ is invertible} \Longleftrightarrow I_n-BA \text{ is invertible},
    $$
    where $ I_n \in M_n(\mathbb{C}) $ is identity matrix.
\end{theorem}

The key of the proof of above theorem is the following equation:
\begin{equation} \label{equ1}
    (I_n - BA)(I + B (I - AB)^{-1} A) = I.
\end{equation}
Now a problem is coming, how can we find \eqref{equ1}. Next I will show a intersting idea from StackExchange.

Absolutely, our goal is using $ I_n-AB $ and $ (I_n-AB)^{-1} $ to construct $ {I_n - BA}^{-1} $.
We assume $ (I_n - AB)^{-1} $ and $ (I_n - BA)^{-1} $ are exist and can be expand series, then
\begin{align*}
    (I_n - AB)^{-1} &= I_n + AB + (AB)^2 + \cdots, \\
    (I_n - BA)^{-1} &= I_n + BA + (BA)^2 + \cdots.
\end{align*}
Then
\begin{align*}
    B(I_n - AB)^{-1}A &= BA + (BA)^2 + (BA)^3 + \cdots, \\
                      &= (I_n - BA)^{-1} - I_n.
\end{align*}
From this we get \eqref{equ1}. Of course, this isn't a proof, but the argument is heuristic. Next is a strict proof.

\begin{proof}[Proof of Theorem \ref{thm1}]
    We should only proof, if $ I_n-AB $ is invertible, then $ I_n-BA $ is invertible.
    In fact, assume $ I_n-AB $ is invertible, then
    \begin{align*}
        &(I_n - BA)[I_n + B (I_n - AB)^{-1} A] \\
            &\quad= I_n + B (I_n - AB)^{-1} A - BA - BAB (I_n - AB)^{-1} A \\
            &\quad= I_n - BA + B (I_n - AB) (I_n - AB)^{-1} A = I_n.
    \end{align*}
    Similar we have
    $$
        [I_n + B (I_n - AB)^{-1} A](I_n - BA) = I_n.
    $$
    So $ I_n-BA $ is invertible. This complete the proof of Theorem \ref{thm1}.
\end{proof}

\end{CJK*}
\end{document} 

