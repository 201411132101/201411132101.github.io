\documentclass[a4paper,11pt]{article}
\usepackage{graphicx}
\usepackage{subfigure}
\usepackage{geometry}
\usepackage{framed}
\usepackage{lipsum}
\usepackage{color}
\usepackage{CJKutf8}
\usepackage{amsmath}
\usepackage{amssymb}
\usepackage{amsthm}
\usepackage{multirow}
\usepackage{titlesec}
\usepackage{enumerate}
\usepackage{mathrsfs}


% 定义各种环境
\newtheorem{theorem}{Theorem}[section]
\newtheorem{lemma}[theorem]{Lemma}
\newtheorem{corollary}[theorem]{Corollary}
\newtheorem{proposition}[theorem]{Proposition}
\newtheorem{example}[theorem]{Example}
\theoremstyle{definition}
\newtheorem{remark}[theorem]{Remark}
\newtheorem{definition}[theorem]{Definition}
\newtheorem{assumption}[theorem]{Assumption}

\def \supp {\mathop\mathrm{\,supp\,}}

% 设置 proof 的格式
% \renewcommand{\proofname}{\emph{Proof}}

% 基本信息
\title{Muckenhoupt Weights}

\begin{document}
\begin{CJK*}{UTF8}{gbsn}

\maketitle

如下内容来自 \cite{g14}.

\begin{definition}[Definition 7.1.1]
    设函数 $ w: \mathbb{R}^n \to [0, \infty) $ 满足对 a.e. $ x \in \mathbb{R}^n $, $ w(x) > 0 $.
    称 $ w \in A_1(\mathbb{R}^n) $, 若对 a.e. $ x \in \mathbb{R}^n $,
    $$
        M(w)(x) \leq C w(x).
    $$
    对于 $ w \in A_1(\mathbb{R}^n) $, 称
    $$
        [w]_{A_1} := \sup_{\text{ cube } Q \subset \mathbb{R}^n} 
            \left[ \frac{1}{|Q|} \int_Q w(t) dt \right] \| w^{-1} \|_{L^\infty (Q)}
    $$
    为 $ w $ 的 $ A_1 $ Muckenhoupt 特征常数.
\end{definition}

\begin{definition}[Definition 7.1.3]
    设 $ p \in (1, \infty) $, $ w: \mathbb{R}^n \to [0, \infty) $ 是局部可积函数 
    且对 a.e. $ x \in \mathbb{R}^n $, $ w(x) > 0 $.
    称 $ w \in A_p(\mathbb{R}^n) $, 若存在正常数 $ C $ 使得对任意方体 $ Q \in \mathbb{R}^n $,
    $$
        \left[ \frac{1}{|Q|} \int_Q w(t) dt \right] 
            \left\{ \frac{1}{|Q|} \int_Q [w(t)]^{-\frac{1}{p-1}}dt \right\}^{p-1} \leq C.
    $$
    对于 $ w \in A_p(\mathbb{R}^n) $, 称
    $$
        [w]_{A_p} := \sup_{\text{ cube } Q \subset \mathbb{R}^n} 
            \left[ \frac{1}{|Q|} \int_Q w(t) dt \right] 
            \left\{ \frac{1}{|Q|} \int_Q [w(t)]^{-\frac{1}{p-1}}dt \right\}^{p-1}
    $$
    为 $ w $ 的 $ A_p $ Muckenhoupt 特征常数.
\end{definition}

\begin{lemma} \label{lemma}
    设 $ f \in L^1_{loc}(\mathbb{R}^n) $ 且 $ \alpha \in [1, \infty) $.
    对任意方体 $ Q \subset \mathbb{R}^n $,
    $$
        \frac{1}{|Q|} \int_Q |f(t)| dt 
            \leq \left\{ \frac{1}{|Q|} \int_Q |f(t)|^\alpha dt \right\}^{\frac{1}{\alpha}}.
    $$
\end{lemma}

\begin{proof}
    对任意方体 $ Q \subset \mathbb{R}^n $, 由 H\"older 不等式知
    \begin{align*}
        \frac{1}{|Q|} \int_Q |f(t)| dt 
            &\leq \frac{1}{|Q|} \left[ \int_Q |f(t)|^\alpha dt \right]^{\frac{1}{\alpha}}
                \left( \int_Q 1^{\alpha'} dt \right)^{\frac{1}{\alpha'}}  \\
            &= \left[ \int_Q |f(t)|^\alpha dt \right]^{\frac{1}{\alpha}} |Q|^{\frac{1}{\alpha'} - 1} 
            = \left\{ \frac{1}{|Q|} \int_Q |f(t)|^\alpha dt \right\}^{\frac{1}{\alpha}}.
    \end{align*}
\end{proof}

\begin{theorem}[习题 7.1.10]
    设 $ p_1, p_2 \in [1, \infty) $ 且 $ w_1 \in A_{p_1}(\mathbb{R}^n), w_2 \in A_{p_2}(\mathbb{R}^n) $, 则
    $$
        [w_1 + w_2]_{A_p} \leq [w_1]_{A_{p_1}} + [w_2]_{A_{p_2}}.
    $$
    其中 $ p = \max(p_1, p_2) $.
\end{theorem}

\begin{proof}
    Case 1), $ p_1 = p_2 = 1 $. 此时 $ p = \max(p_1, p_2) = 1 $,
    \begin{align*}
        [w_1 + w_2]_{A_1} 
            &= \sup_{\text{ cube } Q \subset \mathbb{R}^n} 
                \left\{ \frac{1}{|Q|} \int_Q [w_1(t) + w_2(t)] dt \right\} 
                \left\| [w_1(t) + w_2(t)]^{-1} \right\|_{L^\infty (Q)} \\
            &\leq \sup_{\text{ cube } Q \subset \mathbb{R}^n} 
                \left[ \frac{1}{|Q|} \int_Q w_1(t) dt \right]
                \left\| [w_1(t) + w_2(t)]^{-1} \right\|_{L^\infty (Q)} \\
                &\quad+ \sup_{\text{ cube } Q \subset \mathbb{R}^n} 
                    \left[ \frac{1}{|Q|} \int_Q w_2(t) dt \right]
                    \left\| [w_1(t) + w_2(t)]^{-1} \right\|_{L^\infty (Q)} \\
            &\leq \sup_{\text{ cube } Q \subset \mathbb{R}^n} 
                \left[ \frac{1}{|Q|} \int_Q w_1(t) dt \right]
                \left\| [w_1(t)]^{-1} \right\|_{L^\infty (Q)} \\
                &\quad+ \sup_{\text{ cube } Q \subset \mathbb{R}^n} 
                    \left[ \frac{1}{|Q|} \int_Q w_2(t) dt \right]
                    \left\| [w_2(t)]^{-1} \right\|_{L^\infty (Q)} \\
            &= [w_1]_{A_1} + [w_2]_{A_1}.
    \end{align*}
    
    Case 2), $ p_1 = 1 $, $ p_2 \in (1, \infty) $. 此时 $ p = \max(p_1, p_2) = p_2 $,
    \begin{align*}
        [w_1 + w_2]_{A_{p_2}} 
            &= \sup_{\text{ cube } Q \subset \mathbb{R}^n} 
                \left\{ \frac{1}{|Q|} \int_Q [w_1(t) + w_2(t)] dt \right\}
                \left\{ \frac{1}{|Q|} \int_Q [w_1(t) + w_2(t)]^{-\frac{1}{p_2 - 1}}dt \right\}^{p_2 - 1} \\
            &\leq \sup_{\text{ cube } Q \subset \mathbb{R}^n} 
                \left[ \frac{1}{|Q|} \int_Q w_1(t)  dt \right]
                \left\{ \frac{1}{|Q|} \int_Q [w_1(t) + w_2(t)]^{-\frac{1}{p_2 - 1}}dt \right\}^{p_2 - 1} \\
                &\quad+ \sup_{\text{ cube } Q \subset \mathbb{R}^n} 
                    \left[ \frac{1}{|Q|} \int_Q w_2(t)  dt \right] 
                    \left\{ \frac{1}{|Q|} \int_Q [w_1(t) + w_2(t)]^{-\frac{1}{p_2 - 1}}dt \right\}^{p_2 - 1} \\
            &\leq \sup_{\text{ cube } Q \subset \mathbb{R}^n} 
                \left[ \frac{1}{|Q|} \int_Q w_1(t)  dt \right]
                \left\| [w_1(t)]^{-1} \right\|_{L^\infty (Q)} \\
                &\quad+ \sup_{\text{ cube } Q \subset \mathbb{R}^n} 
                    \left[ \frac{1}{|Q|} \int_Q w_2(t)  dt \right] 
                    \left\{ \frac{1}{|Q|} \int_Q [w_2(t)]^{-\frac{1}{p_2 - 1}}dt \right\}^{p_2 - 1} \\
            &= [w_1]_{A_1} + [w_2]_{A_{p_2}}.
    \end{align*}
    
    Case 3), $ p_1 \in (1, \infty) $, $ p_2 = 1 $. 类似 Case 2) 可证.
    
    Case 4), $ p_1, p_2 \in (1, \infty) $. 由 Lemma \ref{lemma} 知,
    \begin{align*}
        [w_1 + w_2]_{A_p} 
            &= \sup_{\text{ cube } Q \subset \mathbb{R}^n} 
                \left\{ \frac{1}{|Q|} \int_Q [w_1(t) + w_2(t)] dt \right\}
                \left\{ \frac{1}{|Q|} \int_Q [w_1(t) + w_2(t)]^{-\frac{1}{p - 1}}dt \right\}^{p - 1} \\
            &\leq \sup_{\text{ cube } Q \subset \mathbb{R}^n} 
                \left[ \frac{1}{|Q|} \int_Q w_1(t)  dt \right]
                \left\{ \frac{1}{|Q|} \int_Q [w_1(t) + w_2(t)]^{-\frac{1}{p - 1}}dt \right\}^{p - 1} \\
                &\quad+ \sup_{\text{ cube } Q \subset \mathbb{R}^n} 
                    \left[ \frac{1}{|Q|} \int_Q w_2(t)  dt \right] 
                    \left\{ \frac{1}{|Q|} \int_Q [w_1(t) + w_2(t)]^{-\frac{1}{p - 1}}dt \right\}^{p - 1} \\
            &\leq \sup_{\text{ cube } Q \subset \mathbb{R}^n} 
                \left[ \frac{1}{|Q|} \int_Q w_1(t)  dt \right]
                \left\{ \frac{1}{|Q|} \int_Q [w_1(t)]^{-\frac{1}{p - 1}}dt \right\}^{p - 1} \\
                &\quad+ \sup_{\text{ cube } Q \subset \mathbb{R}^n} 
                    \left[ \frac{1}{|Q|} \int_Q w_2(t)  dt \right] 
                    \left\{ \frac{1}{|Q|} \int_Q [w_2(t)]^{-\frac{1}{p - 1}}dt \right\}^{p - 1} \\
            &\leq \sup_{\text{ cube } Q \subset \mathbb{R}^n} 
                \left[ \frac{1}{|Q|} \int_Q w_1(t)  dt \right]
                \left\{ \frac{1}{|Q|} \int_Q [w_1(t)]^{-\frac{1}{p_1 - 1}}dt \right\}^{p_1 - 1} \\
                &\quad+ \sup_{\text{ cube } Q \subset \mathbb{R}^n} 
                    \left[ \frac{1}{|Q|} \int_Q w_2(t)  dt \right] 
                    \left\{ \frac{1}{|Q|} \int_Q [w_2(t)]^{-\frac{1}{p_2 - 1}}dt \right\}^{p_2 - 1} \\
            &= [w_1]_{A_{p_1}} + [w_2]_{A_{p_2}}.
    \end{align*}
\end{proof}


\begin{thebibliography}{99}

    \bibitem{g14}  L. Grafakos, Classical Fourier Analysis, Third edition, Graduate Texts in Mathe-
    matics, 249, Springer, New York, 2014.
    
\end{thebibliography}

\end{CJK*}
\end{document} 


