\documentclass[a4paper,11pt]{article}
\usepackage{graphicx}
\usepackage{subfigure}
\usepackage{geometry}
\usepackage{framed}
\usepackage{lipsum}
\usepackage{color}
\usepackage{CJKutf8}
\usepackage{amsmath}
\usepackage{amssymb}
\usepackage{amsthm}
\usepackage{multirow}
\usepackage{titlesec}
\usepackage{enumerate}
\usepackage{mathrsfs}


% 定义各种环境
\newtheorem{theorem}{Theorem}[section]
\newtheorem{lemma}[theorem]{Lemma}
\newtheorem{corollary}[theorem]{Corollary}
\newtheorem{proposition}[theorem]{Proposition}
\newtheorem{example}[theorem]{Example}
\theoremstyle{definition}
\newtheorem{remark}[theorem]{Remark}
\newtheorem{definition}[theorem]{Definition}
\newtheorem{assumption}[theorem]{Assumption}

\def \supp {\mathop\mathrm{\,supp\,}}

% 设置 proof 的格式
% \renewcommand{\proofname}{\emph{Proof}}

% 基本信息
\title{齐型空间上的拓扑}

\begin{document}
\begin{CJK*}{UTF8}{gbsn}

\maketitle

\begin{definition} \label{10.21.1}
A \emph{quasi-metric space} $ (\mathcal{X}, d) $ is a non-empty set $ \mathcal{X} $ equipped with a \emph{quasi-metric} $ d $,
namely, a non-negative function defined on $ \mathcal{X} \times \mathcal{X} $, satisfying that, for any $ x, y, z \in \mathcal{X} $,
    \begin{enumerate}[{\rm(i)}]
        \item $ d(x, y) = 0 $ if and only if $ x = y $;
        \item $ d(x, y) = d(y, x) $;
        \item there exists a constant $ A_0 \in [1, \infty) $, independent of $ x $, $ y $,
            and $ z $, such that $ d(x, z) \leq A_0 [d(x, y) + d(y, z)] $.
    \end{enumerate}
\end{definition}

The \emph{ball} $ B $ on $ \mathcal{X} $, centered at $ x_0 \in \mathcal{X} $ with radius $ r \in (0, \infty) $, 
is defined by setting
$$
    B := \left\{x \in \mathcal{X} :\ d(x, x_0) < r \right\} =: B \left( x_0, r \right).
$$

构造齐型空间 $ \mathcal{X} $ 中的拓扑有两种定义方法, 一种是用球来定义开集, 一种是用收敛性来定义闭集.

\begin{definition} \label{def1}
    称 $ E \subset \mathcal{X} $ 是开集, 若对 $ \forall x \in E $, 
    存在 $ \delta \in (0, \infty) $ 使得 $ B(x, \delta) \subset E $.
    称 $ E \subset \mathcal{X} $ 是闭集, 若 $ \mathcal{X} \setminus E $ 是开集.
\end{definition}

\begin{definition} \label{def2}
    称 $ E \subset \mathcal{X} $ 是闭集, 若对 $ \forall x \in E $, 
    存在 $ \{x_k\}_{k \in \mathbb{N}} \subset E $ 使得 $ \lim_{k \to \infty} d(x, x_k) \subset E $.
    称 $ E \subset \mathcal{X} $ 是开集, 若 $ \mathcal{X} \setminus E $ 是闭集.
\end{definition}

\begin{remark}
    Definition \ref{def1} 和 Definition \ref{def2} 是等价的.
\end{remark}

等价性的证明来自 \cite[Theorem 3.1]{rb13}, 但不确定他证明的原创性.

\begin{thebibliography}{99}

    \bibitem{rb13}  G. Rano, and T. Bag. Quasi-metric space and fixed point theorems, 
        International Journal of Mathematics and scientific Computing 3 (2013), 27--31.
    
\end{thebibliography}

\end{CJK*}
\end{document} 


