\documentclass[a4paper,12pt]{article}
\usepackage{graphicx}
\usepackage{subfigure}
\usepackage{geometry}
\usepackage{framed}
\usepackage{lipsum}
\usepackage{color}
\usepackage{CJKutf8}
\usepackage{amsmath}
\usepackage{amssymb}
\usepackage{amsthm}
\usepackage{multirow}
\usepackage{titlesec}
\usepackage{enumerate}
\usepackage{mathrsfs}


% 定义各种环境
\newtheorem{theorem}{Theorem}[section]
\newtheorem{lemma}[theorem]{Lemma}
\newtheorem{corollary}[theorem]{Corollary}
\newtheorem{proposition}[theorem]{Proposition}
\newtheorem{example}[theorem]{Example}
\theoremstyle{definition}
\newtheorem{remark}[theorem]{Remark}
\newtheorem{definition}[theorem]{Definition}
\newtheorem{assumption}[theorem]{Assumption}


% 设置 proof 的格式
% \renewcommand{\proofname}{\emph{Proof}}

% 基本信息
\title{Cauchy-Schwarz Inequality}

\begin{document}
\begin{CJK*}{UTF8}{gbsn}

\maketitle

一般书上讲的都是 Hilbert 空间中的 Cauchy–Schwarz inequality, 张恭庆的书上讲了个更一般的[p.55]. 
而下面的 Cauchy–Schwarz inequality, 是我见过的条件最弱的, 暂时还没找到该定理的出处.

本文中的 $ X $ 均为线性空间.

\begin{definition}
    若二元函数 $ a: X \times X \to \mathbb{C} $ 满足: 
    for any $ \alpha \in \mathbb{C} $ and $ x, y \in X $,
    \begin{enumerate}[(i)]
        \item $ a(\alpha x, y) = \alpha a(x, y) $;
        \item $ a(x, \alpha y) = \overline{\alpha} a(x, y) $.
    \end{enumerate}
    我们称 $ a(\cdot, \cdot) $ 为 $ X $ 上的共轭双线性函数.
\end{definition}

\begin{theorem}[Cauchy–Schwarz inequality]
    设 $ a(\cdot, \cdot) $ 为 $ X $ 上的共轭双线性函数且 for any $ x \in X $,
    $$ 
        a(x, x) \geq 0,
    $$
    then for any $  x, y \in X $,
    \begin{equation} \label{result}
        |a(x, y)|^2 \leq a(x, x) a(y, y).
    \end{equation}
\end{theorem}

\begin{proof}
    For any $  x, y \in X $, if $ a(x, x) = a(y, y) = 0 $, then
    \begin{align*}
        0 &\leq a(x + y, x + y) = a(x, x) + a(y, y) + 2 \text{Re} a(x, y) = 2 \text{Re} a(x, y), \\
        0 &\leq a(x - y, x - y) = a(x, x) + a(y, y) - 2 \text{Re} a(x, y) = - 2 \text{Re} a(x, y), \\
        0 &\leq a(x + iy, x + iy) = a(x, x) + a(y, y) + 2 \text{Im} a(x, y) = 2 \text{Im} a(x, y), \\
        0 &\leq a(x - iy, x + iy) = a(x, x) + a(y, y) - 2 \text{Im} a(x, y) = - 2 \text{Im} a(x, y),
    \end{align*}
    So $ \text{Re} a(x, y) = 0 = \text{Im} a(x, y) $, i.e $ a(x, y) = 0 $. 
    $$ 
        |a(x, y)|^2 = 0 = a(x, x) a(y, y).
    $$
    Otherwise, we assume $ a(y, y) \neq 0 $, let $ \lambda := a(x, y) / a(y, y) $, then
    \begin{align*}
        0 &\leq a(x - \lambda y, x - \lambda y)\\
          &= a(x, x) - \overline{\lambda} a(x, y) - \lambda a(y, x) + |\lambda|^2 a(y, y) \\
          &= a(x, x) - \frac{|a(x, y)|^2}{a(y, y)}.
    \end{align*}
    From this we have
    $$ 
        |a(x, y)|^2 \leq a(x, x) a(y, y).
    $$
    The proof of Cauchy-Schwarz inequality is finished.
\end{proof}

\end{CJK*}
\end{document} 




