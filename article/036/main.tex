\documentclass[a4paper,11pt]{article}
\usepackage{graphicx}
\usepackage{subfigure}
\usepackage{geometry}
\usepackage{framed}
\usepackage{lipsum}
\usepackage{color}
\usepackage{amsmath}
\usepackage{amssymb}
\usepackage{amsthm}
\usepackage{multirow}
\usepackage{titlesec}
\usepackage{enumerate}
\usepackage{mathrsfs}
\usepackage{ctex}


% 定义各种环境
\newtheorem{theorem}{Theorem}[section]
\newtheorem{lemma}[theorem]{Lemma}
\newtheorem{corollary}[theorem]{Corollary}
\newtheorem{proposition}[theorem]{Proposition}
\newtheorem{example}[theorem]{Example}
\theoremstyle{definition}
\newtheorem{remark}[theorem]{Remark}
\newtheorem{definition}[theorem]{Definition}
\newtheorem{assumption}[theorem]{Assumption}

\def \supp {\mathop\mathrm{\,supp\,}}

% 设置 proof 的格式
% \renewcommand{\proofname}{\emph{Proof}}

% 基本信息
\title{$ L^p(\mathbb{R}^n) $ 对偶等式的推广}

\begin{document}


\maketitle

\section{$ L^p(\mathbb{R}^n) $ 的对偶}

讲到 $ L^p(\mathbb{R}^n) $ 的对偶, 那就会提到如下定理.

\begin{theorem}
    设 $ p \in [1, \infty) $. 则对 $ \forall f \in L^p(\mathbb{R}^n) $,
    $$
        \| f \|_{L^p(\mathbb{R}^n)} = \sup_{g \in L^{p'}(\mathbb{R}^n), \| g \|_{L^{p'}(\mathbb{R}^n)} \, = 1} 
            \int_{\mathbb{R}^n} f(x) g(x) \, dx.
    $$
\end{theorem}

事实上, 该定理条件中的 $ f \in L^p(\mathbb{R}^n) $ 可减弱为 $ f $ 是 $ \mathbb{R}^n $ 上的可测函数.

\begin{theorem}[H. Brezis, Functional Analysis, Exercise 4.7] 
    设 $ p \in [1, \infty) $, $ f $ 是 $ \mathbb{R}^n $ 上的可测函数.
    若 $ \forall g \in L^{p'}(\mathbb{R}^n) $, 有 $ fg \in L^1(\mathbb{R}^n) $,
    则 $ f \in L^p(\mathbb{R}^n) $.
\end{theorem}

\begin{corollary}
    设 $ p \in [1, \infty) $, $ f $ 是 $ \mathbb{R}^n $ 上的可测函数. 则
    存在 $ g \in L^{p'}(\mathbb{R}^n) $,$ \| g \|_{L^{p'}(\mathbb{R}^n)} = 1 $ 使得
    $$
        \| f \|_{L^p(\mathbb{R}^n)} = \int_{\mathbb{R}^n} f(x) g(x) \, dx.
    $$
\end{corollary}

另一种推广如下. 要求一致有界, 但可以减弱 $ g $ 的范围.

\begin{theorem} \label{text}
    设 $ p \in [1, \infty) $, $ f $ 是 $ \mathbb{R}^n $ 上的可测函数. 则
    \begin{equation} \label{haha}
        \| f \|_{L^p(\mathbb{R}^n)} = \sup_{g \in C_c^\infty(\mathbb{R}^n), \| g \|_{L^{p'}(\mathbb{R}^n)} = 1} 
            \int_{\mathbb{R}^n} f(x) g(x) \, dx.
    \end{equation}
\end{theorem}

\subsection{实分析角度的证明}

证明思路: $ f \in L^p(\mathbb{R}^n) $ 时, 该定理回到了经典情况, 
故只需证明 $ \| f \|_p = \infty $ 时, 右边也等于无穷. 
即证右边小于无穷时, $ f \in L^p(\mathbb{R}^n) $.
此时先证
$$
    \sup_{g \in L^\infty(\mathbb{R}^n), g \text{有紧支集}, \| g \|_{L^{p'}(\mathbb{R}^n)} = 1} 
        \int_{\mathbb{R}^n} f(x) g(x) \, dx < \infty.
$$
再取一列 $ g_n $ 使得 $ fg_n $ 逼近 $ |f|^p $. 从而证得 $ f \in L^p(\mathbb{R}^n) $.

首先定义卷积核. 对 $ \forall x \in \mathbb{R}^n $,
$$
    \rho(x) := \left\{ \begin{aligned}
        &\exp\left( \frac{1}{|x|^2 - 1} \right), && |x| < 1, \\
        &0,                             && |x| \geq 1,
    \end{aligned}\right.
$$
则 $ \rho \in C_c^\infty(\mathbb{R}^n) $, $ \supp(\rho) = \overline{B(0, 1)} $ 且 $ \rho \geq 0 $. 
对 $ \forall k \in \mathbb{N} $ 和 $ \forall x \in \mathbb{R}^n $, 令
$$
    \rho_k(x) := \frac{k^n}{\| \rho \|_{L^1(\mathbb{R}^n)}} \rho(kx)
        = \left\{ \begin{aligned}
        &\frac{k^n}{\| \rho \|_{L^1(\mathbb{R}^n)}} \exp\left( \frac{1}{|kx|^2 - 1} \right), && |kx| < 1, \\
        &0,                             && |kx| \geq 1,
    \end{aligned}\right.
$$
则 $ \rho_k \in C_c^\infty(\mathbb{R}^n) $, $ \supp(\rho_k) = \overline{B(0, 1/k)} $, $ \rho_k \geq 0 $
且 $ \| \rho_k \|_{L^1(\mathbb{R}^n)} = 1 $. 


\begin{lemma} \label{924}
    设 $ A \subset \mathbb{R}^n $ 为有界闭集, $ A \subset \mathbb{R}^n $ 为闭集, 则 $ A+B $ 为闭集.
\end{lemma}

\begin{proof}
    设 $ \{x_k\}_{k \in \mathbb{N}} \subset A+B $ 在 $ \mathbb{R}^n $ 中收敛到 $ x_0 $.
    则存在 $ \{y_k\}_{k \in \mathbb{N}} \subset A $ 和 $ \{z_k\}_{k \in \mathbb{N}} \subset B $
    使得, 对 $ \forall k \in \mathbb{N} $,
    $$
        x_k = y_k + z_k.
    $$
    由于 $ A $ 为有界闭集, 故可取 $ \{y_{n_k}\}_{k \in \mathbb{N}} $ 为 $ \{y_k\}_{k \in \mathbb{N}} $ 的收敛子列.
    记 $ y_0 := \lim_{k \to \infty} y_{n_k} $, 则 $ y_0 \in A $. 又由 $ B $ 是闭集知
    $$
        \lim_{k \to \infty} z_{n_k} 
            = \lim_{k \to \infty} x_{n_k} - \lim_{k \to \infty} y_{n_k} 
            = x_0 - y_0
            \in B.
    $$
    故
    $$
        x_0 = y_0 + (x_0 - y_0) \in A + B.
    $$
    因此 $ A + B $ 是闭集.
\end{proof}

\begin{remark}
    注意, 两闭集必须有其中之一是有界的, 否则结论不一定成立.
    取 
    $$ 
        A := \left\{ k + \frac{1}{k} \right\}_{k \in \mathbb{N}}, 
        \quad \text{and} \quad 
        B := \mathbb{Z},
    $$ 
    则
    $$
        A + B = \left\{ k + \frac{1}{m}:\ k \in \mathbb{Z},\ m \in \mathbb{N} \right\}
    $$
    不是闭集.
\end{remark}

有了这些准备工作, 现在可以开始证明 Theorem \ref{text} 了.

\begin{proof}[Proof of Theorem \ref{text}]
    当 $ f \in L^p(\mathbb{R}^n) $ 时, \eqref{haha} 是经典的等式, 证明略.
    
    断言, 若
    $$
        \sup_{g \in C_c^\infty(\mathbb{R}^n), \| g \|_{L^{p'}(\mathbb{R}^n)} = 1} 
            \int_{\mathbb{R}^n} f(x) g(x) \, dx =: M < \infty.
    $$
    则 $ f \in L^p(\mathbb{R}^n) $.
    事实上, 当 $ M < \infty $ 时, $ f \in L_{loc}^1(\mathbb{R}) $ 是自动的.
    令 $ g \in L^\infty(\mathbb{R}^n) $, 有紧支集且 $ \| g \|_{L^{p'}(\mathbb{R}^n)} = 1 $.
    对 $ \forall k \in \mathbb{N} $, 设 $ g_k := \rho_k * g $, 则 $ g_k \in C_c^\infty(\mathbb{R}^n) $,
    从而
    $$
        \int_{\mathbb{R}^n} f(x) g_k(x) \, dx \leq M \| g_k \|_{L^{p'}(\mathbb{R}^n)}.
    $$
    因为 $ \lim_{k \to \infty} \| g_k - g \|_{L^1(\mathbb{R}^n)} = 0 $, 由 Riesz 定理知, 
    存在子列 $ g_{n_k} $ 几乎处处收敛到 $ g $. 又由 $ g_k $ 的定义知, 
    $ \|g_k\|_{L^\infty(\mathbb{R}^n)} \leq \|g\|_{L^\infty(\mathbb{R}^n)} $ 
    且由引理 \ref{924} 知,
    \begin{align*}
        \supp(g_k) &= \supp(\rho_k * g) 
                   \subset \overline{\supp(\rho_k) + \supp(g)} \\
                   &= \overline{\overline{B(0, 1/k)} + \supp(g)} 
                   = \overline{B(0, 1/k)} + \supp(g)
                   \subset \overline{B(0, 1)} + \supp(g), 
    \end{align*}
    由此及 $ f \in L^1_{loc}(\mathbb{R}^n) $ 知, 
    对 $ \forall k \in \mathbb{N} $, 在几乎处处意义下
    $$
        |f g_k| \leq \|g_k\|_{L^\infty(\mathbb{R}^n)} |f \mathbf{1}_{\supp(g_k)}|
                \leq \|g\|_{L^\infty(\mathbb{R}^n)} \left|f \mathbf{1}_{\overline{B(0, 1)} + \supp(g)}\right| 
                \in L^1(\mathbb{R}^n).
    $$
    故由控制收敛定理
    \begin{align*}
        \int_{\mathbb{R}^n} f(x) g(x) \, dx
            &= \int_{\mathbb{R}^n} \lim_{k \to \infty} f(x) g_{n_k}(x) \, dx \\
            &= \lim_{k \to \infty} \int_{\mathbb{R}^n} f(x) g_{n_k}(x) \, dx \\
            &\leq \lim_{k \to \infty} M \| g_{n_k} \|_{L^{p'}(\mathbb{R}^n)}
            = M.
    \end{align*}
    从而
    $$
        \sup_{g \in L^\infty(\mathbb{R}^n), g \text{有紧支集}, \| g \|_{L^{p'}(\mathbb{R}^n)} = 1} 
            \int_{\mathbb{R}^n} f(x) g(x) \, dx \leq M.
    $$
    
    若 $ p = 1 $. 对 $ \forall k \in \mathbb{N} $, 令
    $$
        g_k := \mathrm{sign} (f) \mathbf{1}_{B(0, k)},
    $$
    则 $ g_k \in L^\infty(\mathbb{R}^n) $, $ g_k $ 有紧支集且 $ \|g_k\|_{L^\infty(\mathbb{R}^n)} = 1 $, 从而
    \begin{align*}
        \int_{B(0, k)} |f(x)| \, dx
            = \int_{\mathbb{R}^n} f(x) g_k(x) \, dx
            \leq M.
    \end{align*}
    令 $ k \to \infty $, 由 Levi 定理知
    $$
        \| f \|_{L^1(\mathbb{R}^n)} \leq M.
    $$
    若 $ p \in (1, \infty) $. 对 $ \forall k \in \mathbb{N} $, 令
    $$
        g_k := |f|^{p-1} \mathrm{sign}(f) \mathbf{1}_{\{|f| < k\}} \mathbf{1}_{B(0, k)},
    $$
    则 $ g_k \in L^\infty(\mathbb{R}^n) $ 且 $ g_k $ 有紧支集, 从而
    \begin{align*}
        \int_{B(0, k) \cap \{|f| < k\}} |f(x)|^p \, dx
            &= \int_{\mathbb{R}^n} f(x) g_k(x) \, dx
            \leq M \| g \|_{L^{p'}(\mathbb{R}^n)} \\
            &= M \left[ \int_{B(0, k) \cap \{|f| < k\}} |f(x)|^p \, dx \right]^{1/p'},
    \end{align*}
    故
    $$
        \left[ \int_{B(0, k) \cap \{|f| < k\}} |f(x)|^p \, dx \right]^{1/p} \leq M.
    $$
    令 $ k \to \infty $, 由 Levi 定理知
    $$
        \| f \|_{L^p(\mathbb{R}^n)} \leq M.
    $$
    综上, 断言成立.
    
    因此, 当 $ f \notin L^p(\mathbb{R}^n) $ 时,
    $$
        \sup_{g \in C_c^\infty(\mathbb{R}^n), \| g \|_{L^{p'}(\mathbb{R}^n)} = 1} 
            \int_{\mathbb{R}^n} f(x) g(x) \, dx = \infty
    $$
    \eqref{haha} 仍然成立. 至此 Theorem \ref{text} 证毕.
\end{proof}

\section{泛函角度的证明}

Theorem \ref{text} 的第一种证明思路来源于 Brezis 书中 Corollary 4.24 的证明. 同样的思路在张恭庆的泛函分析中也有提到.
实际上直接利用此结论, 从泛函角度能快速证明 Theorem \ref{text}.

\begin{theorem}[H. Brezis, Functional Analysis, Corollary 4.24] \label{h}
    设开集 $ \Omega \subset \mathbb{R}^n $ 且 $ f \in L^1_{loc}(\Omega) $ 满足
    对 $ \forall g \in C_c^\infty(\Omega) $,
    $$
        \int_\Omega f(x) g(x) \, dx = 0.
    $$
    则 $ f $ 在 $ \Omega $ 上几乎处处为 $ 0 $.
\end{theorem}

\begin{proof}
    为了方便, 只证 $ \Omega = \mathbb{R}^n $ 时的情况.
    令 $ g \in L^\infty(\mathbb{R}^n) $ 且有紧支集.
    对 $ \forall k \in \mathbb{N} $, 设 $ g_k := \rho_k * g $, 则 $ g_k \in C_c^\infty(\mathbb{R}^n) $,
    从而
    $$
        \int_{\mathbb{R}^n} f(x) g_k(x) \, dx = 0.
    $$
    因为 $ \lim_{k \to \infty} \| g_k - g \|_{L^1(\mathbb{R}^n)} = 0 $, 由 Riesz 定理知, 
    存在子列 $ g_{n_k} $ 几乎处处收敛到 $ g $. 又由 $ g_k $ 的定义知, 
    $ \|g_k\|_{L^\infty(\mathbb{R}^n)} \leq \|g\|_{L^\infty(\mathbb{R}^n)} $ 
    且 
    \begin{align*}
        \supp(g_k) &= \supp(\rho_k * g) 
                   \subset \overline{\supp(\rho_k) + \supp(g)} \\
                   &= \overline{\overline{B(0, 1/k)} + \supp(g)} 
                   = \overline{B(0, 1/k)} + \supp(g)
                   \subset \overline{B(0, 1)} + \supp(g), 
    \end{align*}
    由此及 $ f \in L^1_{loc}(\mathbb{R}^n) $ 知, 
    对 $ \forall k \in \mathbb{N} $, 在几乎处处意义下
    $$
        |f g_k| \leq \|g_k\|_{L^\infty(\mathbb{R}^n)} |f \mathbf{1}_{\supp(g_k)}|
                \leq \|g\|_{L^\infty(\mathbb{R}^n)} \left|f \mathbf{1}_{\overline{B(0, 1)} + \supp(g)}\right| 
                \in L^1(\mathbb{R}^n).
    $$
    故由控制收敛定理
    \begin{align*}
        \int_{\mathbb{R}^n} f(x) g(x) \, dx
            &= \int_{\mathbb{R}^n} \lim_{k \to \infty} f(x) g_{n_k}(x) \, dx \\
            &= \lim_{k \to \infty} \int_{\mathbb{R}^n} f(x) g_{n_k}(x) \, dx = 0.
    \end{align*}
    
    对 $ \forall k \in \mathbb{N} $, 令
    $$
        g_k := \mathrm{sign} (f) \mathbf{1}_{B(0, k)},
    $$
    则 $ g_k \in L^\infty(\mathbb{R}^n) $ 且有紧支集, 从而
    $$
        \int_{B(0, k)} |f(x)| \, dx
            = \int_{\mathbb{R}^n} f(x) g_k(x) \, dx
            = 0,
    $$
    故对 a.e. $ x \in B(0, k) $, $ f(x) = 0 $.
    再由 $ k $ 的任意性知, 对 a.e. $ x \in \mathbb{R}^n $, $ f(x) = 0 $.
\end{proof}

下面给出 Theorem \ref{text} 的另一种证明.

\begin{proof}[Proof of Theorem \ref{text}]
    当 $ f \in L^p(\mathbb{R}^n) $ 时, \eqref{haha} 是经典的等式, 证明略.
    
    若
    $$
        \sup_{g \in C_c^\infty(\mathbb{R}^n), \| g \|_{L^{p'}(\mathbb{R}^n)} = 1} 
            \int_{\mathbb{R}^n} f(x) g(x) \, dx =: M < \infty.
    $$
    则可定义算子
    $$
        T :\ C_c^\infty(\mathbb{R}^n) \to \mathbb{R},\ g \mapsto \int_{\mathbb{R}^n} f(x) g(x) \, dx.
    $$
    该算子是连续线性泛函, 从而可以沿拓为 $ L^{p'}(\mathbb{R}^n) $ 上的连续泛函 $ \widetilde{T} $.
    对于 $ \widetilde{T} $, 存在 $ \tilde{f} \in L^p(\mathbb{R}^n) $ 使得 
    $$
        \widetilde{T}(g) = \int_{\mathbb{R}^n} \tilde{f}(x) g(x) \, dx, \quad \forall g \in L^{p'}(\mathbb{R}^n).
    $$
    对 $ \forall g \in C_c^\infty(\mathbb{R}^n) $,
    $$
        \int_{\mathbb{R}^n} f(x) g(x) \, dx = T(g) = \widetilde{T}(g) = \int_{\mathbb{R}^n} \tilde{f}(x) g(x) \, dx,
    $$
    即
    $$
        \int_{\mathbb{R}^n} \left[f(x) - \tilde{f}(x)\right] g(x) \, dx = 0.
    $$
    由此及 Theorem \ref{h} 知, $ f $ 和 $ \tilde{f} $ 几乎处处相等, 故 $ f \in L^p(\mathbb{R}^n) $.
\end{proof}



\end{document} 









