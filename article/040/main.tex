\documentclass[a4paper,11pt]{article}
\usepackage{graphicx}
\usepackage{subfigure}
\usepackage{geometry}
\usepackage{framed}
\usepackage{lipsum}
\usepackage{color}
\usepackage{CJKutf8}
\usepackage{amsmath}
\usepackage{amssymb}
\usepackage{amsthm}
\usepackage{multirow}
\usepackage{titlesec}
\usepackage{enumerate}
\usepackage{mathrsfs}


% 定义各种环境
\newtheorem{theorem}{Theorem}[section]
\newtheorem{lemma}[theorem]{Lemma}
\newtheorem{corollary}[theorem]{Corollary}
\newtheorem{proposition}[theorem]{Proposition}
\newtheorem{example}[theorem]{Example}
\theoremstyle{definition}
\newtheorem{remark}[theorem]{Remark}
\newtheorem{definition}[theorem]{Definition}
\newtheorem{assumption}[theorem]{Assumption}

\def \supp {\mathop\mathrm{\,supp\,}}

% 设置 proof 的格式
% \renewcommand{\proofname}{\emph{Proof}}

% 基本信息
\title{共轭空间}

\begin{document}
\begin{CJK*}{UTF8}{gbsn}

\maketitle

\section{共轭算子(Adjoint operator)}

共轭算子是个非常有用的工具.

\begin{definition}
    设 $ \mathcal{X},\ \mathcal{Y} $ 是线性赋范空间, $ T \in \mathscr{L}(\mathcal{X}, \mathcal{Y}) $.
    若 $ T^* \in \mathscr{L}(\mathcal{Y}^*, \mathcal{X}^*) $ 满足: 
    对 $ \forall x \in \mathcal{X} $ 和 $ \forall f \in \mathcal{Y}^* $,
    $$
        \langle f, Tx \rangle = \langle T^* f, x \rangle,
    $$
    则称 $ T^* $ 是 $ T $ 的共轭算子.
\end{definition}

\begin{remark}
    注意, 共轭算子是一定存在的, 且 
    $ \| T^* \|_{\mathscr{L}(\mathcal{Y}^*, \mathcal{X}^*)} = \| T \|_{\mathscr{L}(\mathcal{X}, \mathcal{Y})} $.
\end{remark}

\begin{proof}
    先证存在性. 对 $ \forall f \in \mathcal{Y}^* $, 定义
    $$
        g_f :\ \mathcal{X} \to \mathbb{K},\ x \mapsto \langle f, Tx \rangle,
    $$
    则易证 $ g_f \in \mathcal{X}^* $. 令
    $$
        T^* :\ \mathcal{Y}^* \to \mathcal{X}^*,\ f \to g_f,
    $$
    则易证 $ T^* \in \mathscr{L}(\mathcal{Y}^*, \mathcal{X}^*) $且
    对 $ \forall x \in \mathcal{X} $ 和 $ \forall f \in \mathcal{Y}^* $,
    $$
        \langle f, Tx \rangle = \langle T^* f, x \rangle.
    $$
    
    再证 $ \| T^* \|_{\mathscr{L}(\mathcal{Y}^*, \mathcal{X}^*)} = \| T \|_{\mathscr{L}(\mathcal{X}, \mathcal{Y})} $.
    一方面, 
    \begin{align*}
        \| T^* \|_{\mathscr{L}(\mathcal{Y}^*, \mathcal{X}^*)}
            &= \sup_{\| f \|_{\mathcal{Y}^*} = 1}  \| T^* f \|_{\mathcal{X}^*}
            = \sup_{\| f \|_{\mathcal{Y}^*} = 1} \sup_{\| x \|_\mathcal{X} = 1} |\langle T^* f, x \rangle| \\
            &= \sup_{\| f \|_{\mathcal{Y}^*} = 1} \sup_{\| x \|_\mathcal{X} = 1} |\langle f, Tx \rangle|
            \leq \sup_{\| x \|_\mathcal{X} = 1} \| Tx \|_\mathcal{Y}
            \leq \| T \|_{\mathscr{L}(\mathcal{X}, \mathcal{Y})}.
    \end{align*}
    另一方面, 由 Hahn-Banach 定理推论(Corollary \ref{haha})知
    \begin{align*}
        \| T \|_{\mathscr{L}(\mathcal{X}, \mathcal{Y})}
            &= \sup_{\| x \|_\mathcal{X} = 1}  \| T x \|_\mathcal{Y}
            \overset{Corollary \, \ref{haha}}{=} 
                \sup_{\| x \|_\mathcal{X} = 1} \max_{\| f \|_{\mathcal{Y}^*} = 1}  f(T x) \\
            &= \sup_{\| x \|_\mathcal{X} = 1} \max_{\| f \|_{\mathcal{Y}^*} = 1} \langle T^* f, x \rangle
            \leq \max_{\| f \|_{\mathcal{Y}^*} = 1} \| T^* f \|_{\mathcal{X}^*}
            \leq \| T^* \|_{\mathscr{L}(\mathcal{Y}^*, \mathcal{X}^*)}.
    \end{align*}
    因此 $ \| T^* \|_{\mathscr{L}(\mathcal{Y}^*, \mathcal{X}^*)} = \| T \|_{\mathscr{L}(\mathcal{X}, \mathcal{Y})} $.
\end{proof}

共轭算子会保持原来算子的性质.

\begin{proposition} \label{1}
    设 $ \mathcal{X},\ \mathcal{Y} $ 是 Banach 空间, $ T \in \mathscr{L}(\mathcal{X}, \mathcal{Y}) $. 
    若 $ T $ 是双射, 则 $ T^* $ 也是双射.
\end{proposition}

\begin{proof} 
    由开映射定理知, $ T^{-1} \in \mathscr{L}(\mathcal{Y}, \mathcal{X}) $, 因此 $ (T^{-1})^* $ 存在.
    对 $ \forall x^* \in \mathcal{X}^* $, 令 $ y^* := (T^{-1})^* (x^*) $,
    则 $ y^* \in \mathcal{Y}^* $ 且对 $ \forall x \in \mathcal{X} $,
    $$
        \langle T^* (y^*), x \rangle 
            = \langle y^*, T x \rangle
            = \langle (T^{-1})^* (x^*), T x \rangle 
            = \langle x^*, x \rangle,
    $$
    即 $ T^* (y^*) = x^* $. 故 $ T^* $ 是满射.
    对 $ \forall y_1^*,\ y_2^* \in \mathcal{Y}^* $, 若 $ T^* (y_1^*) = T^* (y_2^*) $, 
    则对 $ \forall x \in \mathcal{X} $,
    $$
        \langle y_1^*, T x \rangle 
            = \langle T^* (y_1^*), x \rangle 
            = \langle T^* (y_2^*), x \rangle 
            = \langle y_2^*, T x \rangle 
    $$
    由此及 $ T $ 是满射知, $ y_1^* = y_2^* $. 故 $ T^* $ 是单射.
\end{proof}

\begin{remark}
    反过来, 若 $ T^* $ 是双射, 能否推出 $ T $ 是双射? 似乎是对的, Rudin 的书中有零散的结论.
\end{remark}

\begin{proposition} \label{5}
    设 $ \mathcal{X},\ \mathcal{Y} $ 是 Banach 空间, $ T \in \mathscr{L}(\mathcal{X}, \mathcal{Y}) $. 
    若 $ T $ 是线性等距同构映射(满的等距映射), 则 $ T^* $ 也是线性等距同构映射.
\end{proposition}

\begin{proof}
     $ T $ 是线性等距同构映射, 故 $ T $ 是双射, 由此及 Proposition \ref{1} 知, $ T^* $ 是双射.
    又对 $ \forall y^* \in \mathcal{Y}^* $,
    \begin{align*}
        \| T^* y^* \|_{\mathcal{X}^*}
            = \sup_{\| x \|_\mathcal{X} = 1} |\langle T^* y^*, x \rangle|
            = \sup_{\| x \|_\mathcal{X} = 1} |\langle  y^*, T x \rangle|
            = \sup_{\| y \|_\mathcal{Y} = 1} |\langle  y^*, y \rangle|
            = \| y^* \|_{\mathcal{Y}^*}.
    \end{align*}
    故 $ T^* $ 也是线性等距同构映射.
\end{proof}

\section{自反空间(Reflexive Spaces)}

\begin{definition}
    设 $ \mathcal{X} $ 是线性赋范空间. 
    称 $ \mathcal{X}^* := \mathscr{L}(\mathcal{X}, \mathbb{K}) $ 为 $ \mathcal{X} $ 的共轭空间, 
    称 $ \mathcal{X}^{**} := (\mathcal{X}^*)^* $ 为 $ \mathcal{X} $ 的第二共轭空间, 
\end{definition}

\begin{definition}
    设 $ \mathcal{X} $ 是线性赋范空间. 
    对 $ \forall x \in \mathcal{X} $, 
    $$
        X :\ \mathcal{X}^* \to \mathbb{K} ,\ f \mapsto \langle f, x \rangle
    $$
    是 $ \mathcal{X}^{**} $ 中的元素, 称 
    $$
        J_\mathcal{X} :\ \mathcal{X} \to \mathcal{X}^{**} ,\ x \mapsto X
    $$ 
    为 $ \mathcal{X} $ 的自然映射.
\end{definition}

\begin{definition}
    设 $ \mathcal{X} $ 是线性赋范空间. 
    称 $ \mathcal{X} $ 是自反空间, 若自然映射 $ J_\mathcal{X} $ 是满射.
\end{definition}

\begin{corollary}[H. Brezis, Functional Analysis, Corollary 1.4] \label{haha}
    设 $ \mathcal{X} $ 是线性赋范空间. 则对 $ \forall x \in \mathcal{X} $,
    $$
        \| x \|_\mathcal{X} = \sup_{ \| f \|_{\mathcal{X}^*} = 1} |f(x)| 
                            = \max_{ \| f \|_{\mathcal{X}^*} = 1} |f(x)|.
    $$
\end{corollary}

\begin{proposition}
    $ J_\mathcal{X} $ 是线性等距映射.
\end{proposition}

\begin{proof}
    $ J_\mathcal{X} $ 线性显然. 又由 Hahn-Banach 定理推论(Corollary \ref{haha})知
    $$
        \| J_\mathcal{X} x \|_{\mathcal{X}^{**}} 
            = \sup_{ \| f \|_{\mathcal{X}^*} = 1} |\langle J_\mathcal{X} x, f \rangle|
            = \sup_{ \| f \|_{\mathcal{X}^*} = 1} |f(x)|
            \overset{Corollary \, \ref{haha}}{=} \| x \|_\mathcal{X}.
    $$
    故 $ J_\mathcal{X} $ 是线性等距映射.
\end{proof}

\begin{remark}
    若线性赋范空间 $ \mathcal{X} $ 是自反空间, 则 $ J_\mathcal{X} $ 是满的线性等距映射,
    故 $ \mathcal{X} $ 和 $ \mathcal{X}^{**} $ 等距同构. 
    注意到, $ \mathcal{X}^{**} $ 是完备的, 因此自反空间必定完备.
    
    一个很自然的问题是, 如果 $ \mathcal{X} $ 和 $ \mathcal{X}^{**} $ 等距同构, 
    能否推出 $ \mathcal{X} $ 自反呢? 答案是否定的, James 在 \cite{j51} 中给出了反例.
\end{remark}

下面引入另外两种拓扑, 弱收敛和 *-弱收敛.

\begin{definition}
    设 $ \mathcal{X} $ 是线性赋范空间, $ \{x_n\}_{n = 0}^\infty \subset \mathcal{X} $. 
    若对 $ \forall f \in \mathcal{X}^* $, 
    $$ 
        \lim_{n \to \infty} f(x_n) = f(x_0),
    $$
    则称 $ x_n $ 弱收敛到 $ x_0 $, 记为 $ x_n \rightharpoonup x_0 $.
\end{definition}

\begin{definition}
    设 $ \mathcal{X} $ 是线性赋范空间, $ \{f_n\}_{n = 0}^\infty \subset \mathcal{X}^* $. 
    若对 $ \forall x \in \mathcal{X} $, 
    $$ 
        \lim_{n \to \infty} f_n(x) = f_0(x),
    $$
    则称 $ f_n $ *-弱收敛到 $ f_0 $, 记为 $ w^*-\lim_{n \to \infty} f_n = f_0 $.
\end{definition}

自反空间有些有意思的性质, 比如:

\begin{theorem}
    设 $ \mathcal{X} $ 是自反空间, $ E \subset \mathcal{X} $. 则
    $ E $ 弱列紧当且仅当 $ E $ 有界.
\end{theorem}

\begin{remark}
    "$ E $ 弱列紧 $ \Longrightarrow  E $ 有界" 只需要 $ E $ 是线性赋范空间即可.
\end{remark}

\begin{theorem}[H. Brezis, Functional Analysis, Proposition 3.20] \label{2}
    自反空间的闭线性子空间也自反.
\end{theorem}

\begin{lemma} \label{3}
    等距同构的两线性赋范空间, 自反性相同.
\end{lemma}

\begin{proof}
    设 $ \mathcal{X} $ 和 $ \mathcal{Y} $ 是等距同构的线性赋范空间, 
    则存在 $ \mathcal{X} $ 到 $ \mathcal{Y} $ 的线性等距同构映射 $ \varphi $.
    由 Proposition \ref{5} 知, $ \varphi^{**} $ 也是线性等距同构映射.
    
    设 $ \mathcal{X} $ 自反, 则 $ J_\mathcal{X} $ 是满射, 从而 $ J_\mathcal{X} $ 是双射. 
    下证 $ \mathcal{Y} $ 自反.
    事实上, 对 $ \forall y^{**} \in \mathcal{Y}^{**} $, 
    令 $ y := \varphi( (J_\mathcal{X})^{-1} ( (\varphi^{**})^{-1} (y^{**})) ) $, 
    则 $ y \in \mathcal{Y} $ 且对 $ \forall y^* \in \mathcal{Y}^* $,
    \begin{align*}
        \langle J_\mathcal{Y} y, y^* \rangle
            &= \langle y^*, y \rangle
            = \langle y^*, \varphi( (J_\mathcal{X})^{-1} ( (\varphi^{**})^{-1} (y^{**})) ) \rangle \\
            &= \langle \varphi^* (y^*), (J_\mathcal{X})^{-1} ( (\varphi^{**})^{-1} (y^{**})) \rangle
            = \langle  (\varphi^{**})^{-1} (y^{**}), \varphi^* (y^*) \rangle 
            = \langle y^{**}, y^* \rangle,
    \end{align*}
    即 $ J_\mathcal{Y} y = y^{**} $. 故 $ J_\mathcal{Y} $ 是满射, 从而 $ \mathcal{Y} $ 自反.
\end{proof}

下面给出自反的等价特征.

\begin{theorem}
    设 $ \mathcal{X} $ 是 Banach 空间. 则以下叙述等价:
    \begin{enumerate}[{\rm(i)}]
        \item $ \mathcal{X} $ 自反;
        \item $ \mathcal{X}^* $ 自反;
        \item 单位球弱列紧.
    \end{enumerate}
\end{theorem}

\begin{proof}
    先证 "(i) $\Longrightarrow$ (ii)". 对 $ \forall x^{***} \in \mathcal{X}^{***} $,
    令 $ x^* := (J_\mathcal{X})^* (x^{***}) $, 
    则 $ x^* \in \mathcal{X}^* $ 且对 $ \forall x^{**} \in \mathcal{X}^{**} $,
    \begin{align*}
        \langle J_{\mathcal{X}^*} x^*, x^{**} \rangle
            &= \langle x^{**}, x^* \rangle
            = \langle x^{**}, x^* \rangle
            = \langle J_\mathcal{X} ((J_\mathcal{X})^{-1} x^{**}), x^* \rangle \\
            &= \langle x^*, (J_\mathcal{X})^{-1} x^{**} \rangle
            = \langle (J_\mathcal{X})^* (x^{***}), (J_\mathcal{X})^{-1} x^{**} \rangle
            = \langle x^{***}, x^{**} \rangle,
    \end{align*}
    即 $ J_{\mathcal{X}^*} x^* = x^{***} $. 故 $ J_{\mathcal{X}^*} $ 是满射, 从而 $ \mathcal{X}^* $ 自反.
    
    再证 "(ii) $\Longrightarrow$ (i)". 
    因 $ \mathcal{X} $ 完备, 故 $ J_\mathcal{X}(\mathcal{X}) $ 是 $ \mathcal{X}^{**} $ 的闭线性子空间.
    又因为 $ \mathcal{X}^* $ 自反和 "(i) $\Longrightarrow$ (ii)" 知, $ \mathcal{X}^{**} $ 自反.
    进一步由 Theorem \ref{2} 知, $ J_\mathcal{X}(\mathcal{X}) $ 自反. 
    而 $ J_\mathcal{X}(\mathcal{X}) $ 与 $ \mathcal{X} $ 等距同构, 故由 Lemma \ref{3} 知, $ \mathcal{X} $ 自反.
    
     "(i) $\Longrightarrow$ (iii)" 是经典结论.
     下证 "(iii) $\Longrightarrow$ (i)". 
     证明非常复杂, 需要弄清楚拓扑的构造.
     思路是证 $ J_\mathcal{X}(B_\mathcal{X}) = B_{\mathcal{X}^{**}} $, 
     从而有 $ J_\mathcal{X}(\mathcal{X}) = \mathcal{X}^{**} $.
\end{proof}

\begin{thebibliography}{99}
    \bibitem{j51} R. C. James, A non-reflexive Banach space isometric with its second conjugate space,
     Proc. Nat. Acad. Sci. U.S.A. 37 (1951), 174--177.
\end{thebibliography}

\end{CJK*}
\end{document} 









