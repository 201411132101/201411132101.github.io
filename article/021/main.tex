\documentclass[a4paper,12pt]{article}
\usepackage{graphicx}
\usepackage{subfigure}
\usepackage{geometry}
\usepackage{framed}
\usepackage{lipsum}
\usepackage{color}
\usepackage{CJKutf8}
\usepackage{amsmath}
\usepackage{amssymb}
\usepackage{amsthm}
\usepackage{multirow}
\usepackage{titlesec}
\usepackage{enumerate}
\usepackage{mathrsfs}


% 定义各种环境
\newtheorem{theorem}{Theorem}[section]
\newtheorem{lemma}[theorem]{Lemma}
\newtheorem{corollary}[theorem]{Corollary}
\newtheorem{proposition}[theorem]{Proposition}
\newtheorem{example}[theorem]{Example}
\theoremstyle{definition}
\newtheorem{remark}[theorem]{Remark}
\newtheorem{definition}[theorem]{Definition}
\newtheorem{assumption}[theorem]{Assumption}


% 设置 proof 的格式
% \renewcommand{\proofname}{\emph{Proof}}

% 基本信息
\title{Schwartz Space and Tempered Distributions}

\begin{document}
\begin{CJK*}{UTF8}{gbsn}

\maketitle

在本文中, 
$$ 
    \mathbb{Z}_+ := \{0, 1, 2, \ldots\}.
$$
对 $ \forall \alpha := (\alpha_1, \ldots, \alpha_n) \in \mathbb{Z}_+^n $,
$$
    |\alpha| := \sum_{k = 1}^n |\alpha_k|.
$$
对 $ \forall \alpha := (\alpha_1, \ldots, \alpha_n) \in \mathbb{Z}_+^n $ 
和 $ \forall x := (x_1, \ldots, x_n) \in \mathbb{R}^n $,
$$
    x^\alpha := \prod_{k = 1}^n x_k^{\alpha_k}.
$$ 
$ \mathcal{S} $ 代表 Schwartz 空间.

\subsection{Space of Tempered Distributions Modulo Polynomials}

本 section 的内容来自 \cite[1.1.1]{g14}

首先介绍多项式空间 $ \mathcal{P}(\mathbb{R}^n) $.
\begin{definition}
    $$
        \mathcal{P}(\mathbb{R}^n) 
            := \left\{ \sum_{\alpha \in \mathbb{Z}_+^n, |\alpha| \leq m} c_\alpha x^{\alpha} :\,
                 m \in \mathbb{Z}_+, \text{ for any } \alpha \in \mathbb{Z}_+^n \text{ and } |\alpha| \leq m, c_\alpha \in \mathbb{C} \right\}.
    $$
\end{definition} 

\begin{definition}
    $$
        \mathcal{S}_0(\mathbb{R}^n) 
            := \left\{ \varphi \in \mathcal{S} :\,
                 \text{ for any } \alpha \in \mathbb{Z}_+^n, \int_{\mathbb{R}^n} x^\alpha \varphi(x) dx = 0 \right\}.
    $$
\end{definition} 

\begin{lemma}
    对 $ \forall \alpha \in \mathbb{Z}_+^n, $, $ \partial $ if and only if
    $$
        \int_{\mathbb{R}^n} x^\alpha \varphi(x) dx = 0
    $$
\end{lemma}

\begin{thebibliography}{99}
    
    \bibitem{g14} L. Grafakos, Modern Fourier Analysis, Third edition, 
    Graduate Texts in Mathematics, 249, Springer, New York, 2014.
    
\end{thebibliography}

\end{CJK*}
\end{document} 

