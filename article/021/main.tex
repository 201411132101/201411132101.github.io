\documentclass[a4paper,11pt]{article}
\usepackage{graphicx}
\usepackage{subfigure}
\usepackage{geometry}
\usepackage{framed}
\usepackage{lipsum}
\usepackage{color}
\usepackage{CJKutf8}
\usepackage{amsmath}
\usepackage{amssymb}
\usepackage{amsthm}
\usepackage{multirow}
\usepackage{titlesec}
\usepackage{enumerate}
\usepackage{mathrsfs}


% 定义各种环境
\newtheorem{theorem}{Theorem}[section]
\newtheorem{lemma}[theorem]{Lemma}
\newtheorem{corollary}[theorem]{Corollary}
\newtheorem{proposition}[theorem]{Proposition}
\newtheorem{example}[theorem]{Example}
\theoremstyle{definition}
\newtheorem{remark}[theorem]{Remark}
\newtheorem{definition}[theorem]{Definition}
\newtheorem{assumption}[theorem]{Assumption}

\def \supp {\mathop\mathrm{\,supp\,}}

% 设置 proof 的格式
% \renewcommand{\proofname}{\emph{Proof}}

% 基本信息
\title{Schwartz Space and Tempered Distributions}

\begin{document}
\begin{CJK*}{UTF8}{gbsn}

\maketitle

在本文中, 
\begin{align*}
    \mathbb{N} &:= \{1, 2, \ldots\}, \\
    \mathbb{Z}_+ &:= \{0, 1, 2, \ldots\}.
\end{align*}
对 $ \forall \alpha := (\alpha_1, \ldots, \alpha_n) \in \mathbb{Z}_+^n $,
$$
    |\alpha| := \sum_{k = 1}^n \alpha_k.
$$
对 $ \forall \alpha := (\alpha_1, \ldots, \alpha_n) \in \mathbb{Z}_+^n $ 
和 $ \forall f \in \mathcal{S} $,
$$
    \partial^\alpha f = \frac{\partial^{|\alpha|}}{\partial^{\alpha_1} x_1 \cdots \partial^{\alpha_n} x_n} f.
$$
对 $ \forall \alpha := (\alpha_1, \ldots, \alpha_n) \in \mathbb{Z}_+^n $ 
和 $ \forall x := (x_1, \ldots, x_n) \in \mathbb{R}^n $,
$$
    x^\alpha := \prod_{k = 1}^n x_k^{\alpha_k}.
$$ 
$ \mathcal{S}(\mathbb{R}^n) $ 代表 Schwartz Space.

\subsection{Schwartz Space}

本 section 的内容来自 \cite[2.2]{g14c}.

\begin{proposition}
    设 $ f, g \in \mathcal{S}(\mathbb{R}^n) $, 则 $ f g, f * g \in \mathcal{S}(\mathbb{R}^n) $ 且
    对 $ \forall \alpha \in \mathbb{Z}_+^n $,
    $$
        \partial^\alpha (f * g) =  (\partial^\alpha f) * g = f * (\partial^\alpha g).
    $$
\end{proposition}

\subsection{The Space of Tempered Distributions}

本 section 的内容来自 \cite[2.3]{g14c}.

$ \mathcal{S}'(\mathbb{R}^n) $ 中的傅里叶变换, 导数, 支集.

\begin{definition}[傅里叶变换]
    设 $ T \in \mathcal{S}'(\mathbb{R}^n) $, 对 $ \forall f \in \mathcal{S}(\mathbb{R}^n) $, 定义
    $$
        \widehat{T}(f) := T(\hat{f}).
    $$
\end{definition}

\begin{definition}[导数]
    设 $ T \in \mathcal{S}'(\mathbb{R}^n) $, 对 $ \forall f \in \mathcal{S}(\mathbb{R}^n) $ 和 $ \alpha \in \mathbb{Z}_+^n $, 定义
    $$
        (\partial^\alpha T) (f) := (-1)^{|\alpha|} T (\partial^\alpha f).
    $$
\end{definition}

\begin{remark}
     $ \widehat{T} \in \mathcal{S}'(\mathbb{R}^n) $.
\end{remark}

\begin{definition}[支集]
    设 $ T \in \mathcal{S}'(\mathbb{R}^n) $, 定义 
    $$ 
        \supp T := \bigcap \left\{ K: \text{ for any } f \in \mathcal{S}(\mathbb{R}^n) \text{ and } \supp f \subset \mathbb{R}^n \setminus K, 
            T(f) = 0 \right\}.
    $$
\end{definition}

\begin{remark}
     称 $ T $ is support in $ K $, 
     若对 $ \forall f \in \mathcal{S}(\mathbb{R}^n) $ 且 $ \supp f \subset \mathbb{R}^n \setminus K $, $ T(f) = 0 $.
\end{remark}

\begin{definition}
    设 $ T \in \mathcal{S}'(\mathbb{R}^n) $. 称 $ T $ 在分布意义下与 $ \mathbb{R}^n $ 上的可测函数 $ h $ 是一致的, 
    若对 $ \forall f \in \mathcal{S}(\mathbb{R}^n) $, 有
    $$
        T(f) = \int_{\mathbb{R}^n} h(x) f(x) dx.
    $$
\end{definition}

\begin{proposition}[Proposition 2.4.1]
    设 $ T \in \mathcal{S}'(\mathbb{R}^n) $. 若 $ T $ 支在单点集 $ \{x_0\} $ 上, 则存在
    $ m \in \mathbb{Z}_+ $, $ \{c_\alpha\}_{|\alpha| \leq m} \subset \mathbb{C} $, 使得
    $$
        u = \sum_{\alpha \in \mathbb{Z}_+^n, |\alpha| \leq m} c_\alpha \partial^\alpha \delta_{x_0}.
    $$
\end{proposition}

\subsection{Space of Tempered Distributions Modulo Polynomials}

本 section 的内容来自 \cite[1.1.1]{g14}

首先介绍多项式空间 $ \mathcal{P}(\mathbb{R}^n) $.
\begin{definition}
    $$
        \mathcal{P}(\mathbb{R}^n) 
            := \left\{ \sum_{\alpha \in \mathbb{Z}_+^n, |\alpha| \leq m} c_\alpha x^{\alpha} :\,
                 m \in \mathbb{Z}_+, \text{ for any } \alpha \in \mathbb{Z}_+^n \text{ and } |\alpha| \leq m, c_\alpha \in \mathbb{C} \right\}.
    $$
\end{definition} 


设 $ T_1, T_2 \in \mathcal{S}'(\mathbb{R}^n) $. 称 $ T_1 \equiv T_2 $, 
若存在 $ p \in \mathcal{P}(\mathbb{R}^n) $, 使得 $ T_1 - T_2 $ 在分布意义下与 $ p $ 一致,
则 $ \equiv $ 是个等价关系, 记 $ \mathcal{S}'(\mathbb{R}^n)/ \mathcal{P}(\mathbb{R}^n) $ 为该等价关系诱导的商空间.


\begin{definition}
    $$
        \mathcal{S}_0(\mathbb{R}^n) 
            := \left\{ f \in \mathcal{S}(\mathbb{R}^n) :
                 \text{ for any } \alpha \in \mathbb{Z}_+^n, \int_{\mathbb{R}^n} x^\alpha f(x) dx = 0 \right\}.
    $$
    $ \mathcal{S}_0(\mathbb{R}^n) $ 是 $ \mathcal{S}(\mathbb{R}^n) $ 的子空间.
\end{definition} 



对 $ \forall \alpha \in \mathbb{Z}_+^n $ 和 $ f \in \mathcal{S}(\mathbb{R}^n) $,
由 $ \partial^\alpha (\hat{f}) = ((-2 \pi i x)^\alpha f)^\wedge $ 知, 
$ \partial^\alpha (\hat{f}) (0) = 0 $ 当且仅当
$$
    \int_{\mathbb{R}^n} x^\alpha f(x) dx = 0.
$$
故
$$
    \mathcal{S}_0(\mathbb{R}^n) 
        = \left\{ f \in \mathcal{S} :
             \text{ for any } \alpha \in \mathbb{Z}_+^n, \partial^\alpha (\hat{f}) (0) = 0 \right\}.
$$

\begin{theorem}[Proposition 1.1.3]
    $$
        \mathcal{S}_0'(\mathbb{R}^n) = \mathcal{S}'(\mathbb{R}^n) / \mathcal{P}(\mathbb{R}^n).
    $$
\end{theorem}

\begin{proof}
    先证 $ \mathcal{S}_0'(\mathbb{R}^n) \subset \mathcal{S}'(\mathbb{R}^n) / \mathcal{P}(\mathbb{R}^n) $. 
    对 $ \forall \, T \in \mathcal{S}'(\mathbb{R}^n) $, 定义
    $$
        J(T) := T \big|_{\mathcal{S}_0(\mathbb{R}^n)}.
    $$
    则 $ J $ 是 $ \mathcal{S}'(\mathbb{R}^n) $ 到 $ \mathcal{S}_0'(\mathbb{R}^n) $ 的线性映射. 
    
    断言1: $ \ker J = \mathcal{P}(\mathbb{R}^n) $. 事实上, 
    对 $ \forall p := \sum_{\alpha \in \mathbb{Z}_+^n, |\alpha| \leq m} c_\alpha x^{\alpha} \in  \mathcal{P}(\mathbb{R}^n) $
    和 $ \forall f \in \mathcal{S}_0(\mathbb{R}^n) $,
    $$
        \langle p, f \rangle
            = \int_{\mathbb{R}^n} p(x) f(x) dx
            = \sum_{\alpha \in \mathbb{Z}_+^n, |\alpha| \leq m} c_\alpha \int_{\mathbb{R}^n} x^{\alpha} f(x) dx
            = 0.
    $$
    故 $ p = \theta $ 在 $ \mathcal{S}_0'(\mathbb{R}^n) $ 中成立, 即 $ p \in \ker J $. 因此 
    $$
        \mathcal{P}(\mathbb{R}^n) \subset \ker J.
    $$
    另一方面, 对 $ \forall T \in \ker J $, $ T|_{\mathcal{S}_0(\mathbb{R}^n)} = 0 $, 即
    对 $ \forall f \in \mathcal{S}_0(\mathbb{R}^n) $,
    $$
        \langle T, f \rangle = 0. \eqno{(*)}
    $$
    对 $ \forall g \in C_c^{\infty}(\mathbb{R}^n) $, $\supp g \subset \mathbb{R}^n \setminus \{\vec{0} _n\} $,
    且 $ \alpha \in \mathbb{Z}_+^n $, 
    $$ 
        \partial^\alpha [(\tilde{g}^\vee)^\wedge] (0) = \partial^\alpha (\tilde{g}) (0) = 0, 
    $$
    故 $ \tilde{g}^\vee \in \mathcal{S}_0(\mathbb{R}^n) $, 由此及 $ (*) $ 知,
    $$
        \langle \widehat{T}, g \rangle 
            = \langle \widehat{T}, \tilde{\tilde{g}} \rangle 
            = \langle \widehat{T}, \tilde{g}^{\wedge\wedge} \rangle 
            = \langle T, \tilde{g}^{\wedge\wedge\wedge} \rangle
            = \langle T, \tilde{g}^\vee \rangle 
            = 0,
    $$
    则 $ \widehat{T} $ 支在 $ \{\vec{0}_n\} $ 上. 
    由此及 Proposition 2.4.1 知, $ T \in \mathcal{P}(\mathbb{R}^n) $. 因此
    $$
        \ker J \subset \mathcal{P}(\mathbb{R}^n).
    $$
    从而断言1成立. 
    
    断言2: $ R(J) = \mathcal{S}_0'(\mathbb{R}^n) $.
    事实上, ...
    
    由断言1和断言2知,
    $$
        \mathcal{S}_0'(\mathbb{R}^n) = \mathcal{S}'(\mathbb{R}^n) / \mathcal{P}(\mathbb{R}^n).
    $$
\end{proof}

\begin{thebibliography}{99}
    \bibitem{g14c} L. Grafakos, Classical Fourier Analysis, Third edition, 
    Graduate Texts in Mathematics, 249, Springer, New York, 2014.
        
    \bibitem{g14} L. Grafakos, Modern Fourier Analysis, Third edition, 
    Graduate Texts in Mathematics, 250, Springer, New York, 2014.
    
\end{thebibliography}

\end{CJK*}
\end{document} 

