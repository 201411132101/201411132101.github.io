\documentclass[a4paper,11pt]{article}
\usepackage{graphicx}
\usepackage{subfigure}
\usepackage{geometry}
\usepackage{framed}
\usepackage{lipsum}
\usepackage{color}
\usepackage{CJKutf8}
\usepackage{amsmath}
\usepackage{amssymb}
\usepackage{amsthm}
\usepackage{multirow}
\usepackage{titlesec}
\usepackage{enumerate}
\usepackage{mathrsfs}


% 定义各种环境
\newtheorem{theorem}{Theorem}[section]
\newtheorem{lemma}[theorem]{Lemma}
\newtheorem{corollary}[theorem]{Corollary}
\newtheorem{proposition}[theorem]{Proposition}
\newtheorem{example}[theorem]{Example}
\theoremstyle{definition}
\newtheorem{remark}[theorem]{Remark}
\newtheorem{definition}[theorem]{Definition}
\newtheorem{assumption}[theorem]{Assumption}


% 设置 proof 的格式
% \renewcommand{\proofname}{\emph{Proof}}

% 基本信息
\title{Schwartz Space and Tempered Distributions}

\begin{document}
\begin{CJK*}{UTF8}{gbsn}

\maketitle

在本文中, 
$$ 
    \mathbb{Z}_+ := \{0, 1, 2, \ldots\}.
$$
对 $ \forall \alpha := (\alpha_1, \ldots, \alpha_n) \in \mathbb{Z}_+^n $,
$$
    |\alpha| := \sum_{k = 1}^n \alpha_k.
$$
对 $ \forall \alpha := (\alpha_1, \ldots, \alpha_n) \in \mathbb{Z}_+^n $ 
和 $ \forall f \in \mathcal{S} $,
$$
    \partial^\alpha f = \frac{\partial^{|\alpha|}}{\partial^{\alpha_1} x_1 \cdots \partial^{\alpha_n} x_n} f.
$$
对 $ \forall \alpha := (\alpha_1, \ldots, \alpha_n) \in \mathbb{Z}_+^n $ 
和 $ \forall x := (x_1, \ldots, x_n) \in \mathbb{R}^n $,
$$
    x^\alpha := \prod_{k = 1}^n x_k^{\alpha_k}.
$$ 
$ \mathcal{S} $ 代表 Schwartz 空间.

\subsection{Schwartz Space}

\begin{proposition}
    设 $ f, g \in \mathcal{S}(\mathbb{R}^n) $, 则 $ f g, f * g \in \mathcal{S}(\mathbb{R}^n) $ 且
    对 $ \forall \alpha \in \mathbb{Z}_+^n $,
    $$
        \partial^\alpha (f * g) =  (\partial^\alpha f) * g = f * (\partial^\alpha g).
    $$
\end{proposition}

\subsection{The Space of Tempered Distributions}

\begin{definition}
    设 $ T \in \mathcal{S}'(\mathbb{R}^n) $, 对 $ \forall f \in \mathcal{S}(\mathbb{R}^n) $, 定义
    $$
        \widehat{T}(f) := T(\hat{f}).
    $$
\end{definition}

\begin{remark}
     $ \widehat{T} \in \mathcal{S}'(\mathbb{R}^n) $.
\end{remark}

\begin{definition}
    设 $ T \in \mathcal{S}'(\mathbb{R}^n) $. 称 $ T $ 在分布意义下与 $ \mathbb{R}^n $ 上的可测函数 $ h $ 是一致的, 
    若对 $ \forall f \in \mathcal{S}(\mathbb{R}^n) $, 有
    $$
        T(f) = \int_{\mathbb{R}^n} h(x) f(x) dx.
    $$
\end{definition}

\subsection{Space of Tempered Distributions Modulo Polynomials}

本 section 的内容来自 \cite[1.1.1]{g14}

首先介绍多项式空间 $ \mathcal{P}(\mathbb{R}^n) $.
\begin{definition}
    $$
        \mathcal{P}(\mathbb{R}^n) 
            := \left\{ \sum_{\alpha \in \mathbb{Z}_+^n, |\alpha| \leq m} c_\alpha x^{\alpha} :\,
                 m \in \mathbb{Z}_+, \text{ for any } \alpha \in \mathbb{Z}_+^n \text{ and } |\alpha| \leq m, c_\alpha \in \mathbb{C} \right\}.
    $$
\end{definition} 


设 $ T_1, T_2 \in \mathcal{S}'(\mathbb{R}^n) $. 称 $ T_1 \equiv T_2 $, 
若存在 $ p \in \mathcal{P}(\mathbb{R}^n) $, 使得 $ T_1 - T_2 $ 在分布意义下与 $ p $ 一致,
则 $ \equiv $ 是个等价关系, 记 $ \mathcal{S}'(\mathbb{R}^n)/ \mathcal{P}(\mathbb{R}^n) $ 为该等价关系诱导的商空间.


\begin{definition}
    $$
        \mathcal{S}_0(\mathbb{R}^n) 
            := \left\{ f \in \mathcal{S} :
                 \text{ for any } \alpha \in \mathbb{Z}_+^n, \int_{\mathbb{R}^n} x^\alpha f(x) dx = 0 \right\}.
    $$
    $ \mathcal{S}_0(\mathbb{R}^n) $ 是 $ \mathcal{S}(\mathbb{R}^n) $ 的子空间.
\end{definition} 


对 $ \forall \alpha \in \mathbb{Z}_+^n $ 和 $ f \in \mathcal{S} $, 由 
$$  
    \partial^\alpha (\hat{f}) = ((-2 \pi i x)^\alpha f) \, \hat{}
$$
知 $ \partial^\alpha (\hat{f}) (0) = 0 $ if and only if
$$
    \int_{\mathbb{R}^n} x^\alpha f(x) dx = 0.
$$
故
$$
    \mathcal{S}_0(\mathbb{R}^n) 
        = \left\{ f \in \mathcal{S} :
             \text{ for any } \alpha \in \mathbb{Z}_+^n, \partial^\alpha (\hat{f}) (0) = 0 \right\}.
$$

\begin{thebibliography}{99}
    
    \bibitem{g14} L. Grafakos, Modern Fourier Analysis, Third edition, 
    Graduate Texts in Mathematics, 249, Springer, New York, 2014.
    
\end{thebibliography}

\end{CJK*}
\end{document} 

