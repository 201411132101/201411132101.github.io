\documentclass[a4paper,11pt]{article}
\usepackage{graphicx}
\usepackage{subfigure}
\usepackage{geometry}
\usepackage{framed}
\usepackage{lipsum}
\usepackage{color}
\usepackage{amsmath}
\usepackage{amssymb}
\usepackage{amsthm}
\usepackage{multirow}
\usepackage{titlesec}
\usepackage{enumerate}
\usepackage{mathrsfs}
\usepackage{ctex}


\theoremstyle{definition}
\newtheorem{theorem}{Theorem}
\newtheorem{lemma}[theorem]{Lemma}
\newtheorem{corollary}[theorem]{Corollary}
\newtheorem{proposition}[theorem]{Proposition}
\newtheorem{example}[theorem]{Example}
\newtheorem{remark}[theorem]{Remark}
\newtheorem{definition}[theorem]{Definition}
\newtheorem{assumption}[theorem]{Assumption}



\def \supp {\mathop\mathrm{\,supp\,}}


% 基本信息
\title{弱拓扑及弱$ * $拓扑}

\begin{document}

\maketitle

\section{算子族生成的拓扑}

首先回顾一个经典的结论.

\begin{theorem}[{\cite[p.55]{b11}}] \label{1}
    设 $ \mathcal{X} $ 是集合, $ \{\mathcal{Y}_i\}_{i \in I} $ 是一族拓扑空间, $ \{\varphi_i\}_{i \in I} $ 满足,
    对 $ \forall \, i \in I $, $ \varphi_i : \mathcal{X} \to \mathcal{Y}_i $.
    则存在 $ \mathcal{X} $ 上最小的拓扑 $ \mathcal{J} $ 使得, 对 $ \forall \, i \in I $, $ \varphi_i $ 连续.
\end{theorem}

\begin{definition}
    设记号如定理 \ref{1}, 
    称 $ \mathcal{J} $ 为与 $ \{\varphi_i\}_{i \in I} $ 相关的拓扑.
\end{definition}

\begin{proof}[Proof of Theorem \ref{1}]
    存在性是显然的, 比如可以在 $ \mathcal{X} $ 上定义 discrete topology,
    $$
        \mathcal{J} := \{ \mathcal{X} \text{ 的全部子集} \},
    $$
    则此拓扑满足定理的条件. 关键是证明最小性.
    
    首先注意到, 若要使得, 对 $ \forall \, i \in I $, $ \varphi_i $ 连续, 则
    $$
        \Lambda := \{ \varphi_i^{-1}(\Omega_i) :\ i \in I,\ \Omega_i \text{ 是 $ \mathcal{Y}_i $ 中的开集}  \}
    $$
    必须是拓扑 $ \mathcal{J} $ 的子集. 注意到 $ \Lambda $ 是 $ \mathcal{X} $ 的基,
    故令 $ \mathcal{J} $ 为 $ \Lambda $ 生成的拓扑即可: 对 $ \Lambda $ 取任意有限交得到新集合族 $ \Phi $, 
    再对 $ \Phi $ 取任意并就得到了我们想要的拓扑 $ \mathcal{J} $, 具体证明略.
    
    事实上, $ \Lambda $ 的生成拓扑还有两种等价定义:
    \begin{enumerate}[{\rm(i)}]
        \item 该定义来自 \cite[p.78]{m00}, 
            $$
                \mathcal{J} := \{ U \subset \mathcal{X} :\  
                    \forall \, x \in U,\ \text{there exists a } B \in \Lambda \text{ such that }
                    x \in B \subset U \};
            $$
        \item $ \mathcal{J} := \bigcap \{ \text{包含 } \Lambda \text{ 的拓扑} \} $.
    \end{enumerate}
    定理 \ref{1} 证毕.
\end{proof}

\begin{remark} \label{2}
    对 $ \forall x \in \mathcal{X} $, 可以证明
    $$
        \left\{ \bigcap_{finite} \varphi_i^{-1}(V_i) :\ i \in I,\ V_i \text{是 } \varphi_i(x) \text{ 的邻域} \right\}
    $$
    是 $ x $ 的领域基.
\end{remark}

\begin{proposition}[{\cite[Proposition 3.1]{b11}}]  \label{3}
    设记号如定理 \ref{1}, 
    $ \{x_n\}_{n \in \mathbb{N}} \subset \mathcal{X} $ 且 $ x \in \mathcal{X} $.
    则 $ x_n \overset{\mathcal{J}}{\to} x $, $ \text{as } n \to \infty $,
    当且仅当对 $ \forall \, i \in I $, 
    $ \varphi_i(x_n) \overset{\mathcal{Y}_i}{\to} \varphi_i(x) $, $ \text{as } n \to \infty $.
\end{proposition}

\begin{proof}
    $"\Longrightarrow"$ 是显然的, 下证 $"\Longleftarrow"$. 
    事实上, 对任意 $ x $ 的邻域 $ V $, 存在领域基中的元素 $ U $ 使得 $ U \subset V $. 
    由 Remark \ref{2} 知, $ U $ 可表示为 $ \bigcap_{finite} \varphi_i^{-1}(V_i) $, 
    其中 $ V_i $ 是 $ \varphi_i(x) $ 的邻域. 由于是有限交, 
    故存在 $ N \in \mathbb{N} $ 使得, $ \{x_n\}_{n = N}^\infty \subset U $,
    即 $ x_n \overset{\mathcal{J}}{\to} x $, $ \text{as } n \to \infty $.
    Proposition \ref{3} 证毕.
\end{proof}

\begin{lemma} [{\cite[Proposition 3.1]{b11}}] 
    设 $ \mathcal{Z} $ 是拓扑空间且 $ \psi :\ \mathcal{Z} \to \mathcal{X} $.
    则 $ \psi $ 连续当且仅当对 $ \forall \, i \in I $, $ \varphi_i \circ \psi $ 连续.
\end{lemma}

\section{弱拓扑}

\begin{definition}
    令 $ \mathcal{X}^* := \mathcal{L}(\mathcal{X}, \mathbb{K}) $.
    称 $ \mathcal{X}^* $ 生成的拓扑为 $ \mathcal{X} $ 上的弱拓扑.
\end{definition}

\begin{proposition}
    设 $ \mathcal{X} $ 是 Banach 空间.
    \begin{enumerate}[{\rm(i)}]  
        \item (\cite[Proposition 3.6]{b11}) 若 $ \mathcal{X} $ 是有限维, 则强弱收敛等价;
        \item (\cite[Exercise 3.8]{b11}) 若 $ \mathcal{X} $ 是无穷维, 则无法赋予距离使得, 
            由距离生成的拓扑和弱拓扑一致.
    \end{enumerate}
\end{proposition}

\begin{remark}
    对于 $ \ell^1 $, 强弱收敛等价, 然而范数诱导的拓扑和弱拓扑不一致.
    
    这与直观不矛盾, 因为两个度量的收敛性等价可推出拓扑一致, 但两个拓扑的收敛性等价推不出拓扑一致.
\end{remark}


\begin{theorem} [{\cite[Theorem 3.7]{b11}}]
    设 $ C $ 是 $ \mathcal{X} $ 中的凸集. 
    则 $ C $ 在弱拓扑意义下闭当且仅当 $ C $ 在强拓扑意义下闭.
\end{theorem}


\section{弱$*$拓扑}

设 $ \mathcal{X} $ 为 Banach 空间, 则 $ \mathcal{X}^* $ 上已经有两个拓扑了, 
一个是由 $ \| \cdot \|_{\mathcal{X}^*} $ 生成的强拓扑,
另一个是由 $ \mathcal{X}^{**} $ 生成的弱拓扑,
下面定义第三个拓扑.

\begin{definition}
    设 $ \mathcal{X} $ 为 Banach 空间. 对 $ \forall \, x \in \mathcal{X} $, 定义
    $$
        \varphi_x :\ \mathcal{X}^* \to \mathbb{K},\ f \to f(x).
    $$
    称 $ \{ \varphi_x \}_{x \in \mathcal{X}} $ 生成的拓扑为 $ \mathcal{X}^* $ 上的弱 $ * $ 拓扑.
\end{definition}


\begin{thebibliography}{99}  
	\bibitem{b11}
    H. Brezis, Functional Analysis, Sobolev Spaces and Partial Differential Equations.
    Universitext. Springer, New York,  2011.
    
    \bibitem{m00}
    J. Munkres, Topology, Second edition, Prentice Hall, Inc., Upper Saddle River, NJ, 2000.
\end{thebibliography}

\end{document} 









