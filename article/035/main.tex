\documentclass[a4paper,11pt]{article}
\usepackage{graphicx}
\usepackage{subfigure}
\usepackage{geometry}
\usepackage{framed}
\usepackage{lipsum}
\usepackage{color}
\usepackage{CJKutf8}
\usepackage{amsmath}
\usepackage{amssymb}
\usepackage{amsthm}
\usepackage{multirow}
\usepackage{titlesec}
\usepackage{enumerate}
\usepackage{mathrsfs}


% 定义各种环境
\newtheorem{theorem}{Theorem}[section]
\newtheorem{lemma}[theorem]{Lemma}
\newtheorem{corollary}[theorem]{Corollary}
\newtheorem{proposition}[theorem]{Proposition}
\newtheorem{example}[theorem]{Example}
\theoremstyle{definition}
\newtheorem{remark}[theorem]{Remark}
\newtheorem{definition}[theorem]{Definition}
\newtheorem{assumption}[theorem]{Assumption}

\def \supp {\mathop\mathrm{\,supp\,}}

% 设置 proof 的格式
% \renewcommand{\proofname}{\emph{Proof}}

% 基本信息
\title{范数和拟范数的连续性}

\begin{document}
\begin{CJK*}{UTF8}{gbsn}

\maketitle

范数大家都懂的, 拟范数跟范数差了个三角不等式, 拟范数对应的是拟三角不等式, 即存在正常数 $ K \in (1, \infty) $ 使得
$$
    d(x, y) \leq K[d(x, z) + d(z, y)].
$$

所谓范数和拟范数的连续性, 其实就是想证明范数和拟范数作为算子的连续性.

\begin{theorem}
    设 $ \| \cdot \| $ 是范数, 则
    $$
        \lim_{n \to \infty} \| x_n - x_0 \| = 0 
            \Longrightarrow \lim_{n \to \infty} \| x_n \| = \| x_0 \|.
    $$
\end{theorem}

\begin{proof}
    由
    $$ 
        \left| \| x_n \| - \| x_0 \| \right| \leq  \| x_n - x_0 \|
    $$
    可以看出, 这个结论很显然.
\end{proof}

拟范数跟范数有一定差别, 那么有没有类似的结论呢.
设 $ \| \cdot \| $ 是拟范数, 则是否也有
$$
    \lim_{n \to \infty} \| x_n - x_0 \| = 0 
        \Longrightarrow \lim_{n \to \infty} \| x_n \| = \| x_0 \|.
$$
答案是否定的.

反例: 设 $ K \in (1, \infty) $ 是常数, 且
$$ 
    \| \cdot \| :\ \mathbb{R}^2 \to \mathbb{R},\
        (x, y) \mapsto \left\{\begin{aligned}
        & K|x|,       && y = 0, \\
        & |x| + |y|,  && \text{otherwise}.
        \end{aligned}\right.
$$
取 $ x_n := (1, 1/n) $, $ x_0 := (1, 0) $, 则
$$
    \| x_n - x_0 \| = \frac{1}{n} \to 0, \quad \text{as} \quad n \to \infty.
$$
但是
$$
    \| x_n \| = 1 + \frac{1}{n} \to 1, \quad \text{as} \quad n \to \infty.
$$
而 $ \|x_0\| = K $.

不过对于部分特殊的拟范数, 还是有连续性的, 比如弱 $ L^1 $ 空间. 类似的, 弱 $ L^p $ 应该也是连续的.

\begin{lemma} \label{weak}
    设 $ f,\ \{f_k\}_{k \in \mathbb{N}} \subset L^{1, \infty}(\mathbb{R}) $.
    若
    $$
        \lim_{k \to \infty} \| f_k - f \|_{L^{1, \infty}(\mathbb{R})} = 0,
    $$
    则
    $$
        \lim_{k \to \infty} \| f_k \|_{L^{1, \infty}(\mathbb{R})}
            = \| f \|_{L^{1, \infty}(\mathbb{R})}.
    $$
\end{lemma}

\begin{proof}
    先证
    $$
        \liminf_{k \to \infty} \| f_k \|_{L^{1, \infty}(\mathbb{R})}
            \geq \| f \|_{L^{1, \infty}(\mathbb{R})}.
    $$
    对 $ \forall \lambda \in (0, \infty) $, $ \forall \delta \in (0, 1) $, $ \forall k \in \mathbb{N} $,
    \begin{align*}
        \lambda |\{ x \in \mathbb{R} :\ |f(x)| > \lambda \}|
            &\leq \lambda |\{ x \in \mathbb{R} :\ |f(x) - f_k(x)| > \delta\lambda \}| \\
                &\quad + \lambda |\{ x \in \mathbb{R} :\ |f_k(x)| > (1 - \delta)\lambda \}| \\
            &\leq \frac{1}{\delta} \| f - f_k \|_{L^{1, \infty}(\mathbb{R})}
                + \frac{1}{1 - \delta} \| f_k \|_{L^{1, \infty}(\mathbb{R})},
    \end{align*}
    由 $ \lambda \in (0, \infty) $ 的任意性有,
    $$
        \| f \|_{L^{1, \infty}(\mathbb{R})}
            \leq \frac{1}{\delta} \| f - f_k \|_{L^{1, \infty}(\mathbb{R})}
                + \frac{1}{1 - \delta} \| f_k \|_{L^{1, \infty}(\mathbb{R})},
    $$
    令 $ k \to \infty $ 得,
    $$
        \| f \|_{L^{1, \infty}(\mathbb{R})}
            \leq \frac{1}{1 - \delta} \liminf_{k \to \infty} \| f_k \|_{L^{1, \infty}(\mathbb{R})},
    $$
    再令 $ \delta \to 0^+ $ 知,
    $$
        \| f \|_{L^{1, \infty}(\mathbb{R})}
            \leq \liminf_{k \to \infty} \| f_k \|_{L^{1, \infty}(\mathbb{R})}.
    $$
    再证
    $$
        \limsup_{k \to \infty} \| f_k \|_{L^{1, \infty}(\mathbb{R})}
            \leq \| f \|_{L^{1, \infty}(\mathbb{R})}.
    $$
    对 $ \forall \lambda \in (0, \infty) $, $ \forall \delta \in (0, 1) $, $ \forall k \in \mathbb{N} $,
    \begin{align*}
        \lambda |\{ x \in \mathbb{R} :\ |f_k(x)| > \lambda \}|
            &\leq \lambda |\{ x \in \mathbb{R} :\ |f_k(x) - f(x)| > \delta\lambda \}| \\
                &\quad + \lambda |\{ x \in \mathbb{R} :\ |f(x)| > (1 - \delta)\lambda \}| \\
            &\leq \frac{1}{\delta} \| f_k - f \|_{L^{1, \infty}(\mathbb{R})}
                + \frac{1}{1 - \delta} \| f \|_{L^{1, \infty}(\mathbb{R})},
    \end{align*}
    由 $ \lambda \in (0, \infty) $ 的任意性有,
    $$
        \| f_k \|_{L^{1, \infty}(\mathbb{R})}
            \leq \frac{1}{\delta} \| f_k - f \|_{L^{1, \infty}(\mathbb{R})}
                + \frac{1}{1 - \delta} \| f \|_{L^{1, \infty}(\mathbb{R})},
    $$
    令 $ k \to \infty $ 得,
    $$
        \limsup_{k \to \infty} \| f_k \|_{L^{1, \infty}(\mathbb{R})}
            \leq \frac{1}{1 - \delta} \| f \|_{L^{1, \infty}(\mathbb{R})},
    $$
    再令 $ \delta \to 0^+ $ 知,
    $$
        \limsup_{k \to \infty} \| f_k \|_{L^{1, \infty}(\mathbb{R})}
            \leq \| f \|_{L^{1, \infty}(\mathbb{R})}.
    $$
    综上,
    $$
        \lim_{k \to \infty} \| f_k \|_{L^{1, \infty}(\mathbb{R})}
            = \| f \|_{L^{1, \infty}(\mathbb{R})}.
    $$
\end{proof}

\end{CJK*}
\end{document} 









