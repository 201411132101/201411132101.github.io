\documentclass[a4paper,11pt]{article}
\usepackage{graphicx}
\usepackage{subfigure}
\usepackage{geometry}
\usepackage{framed}
\usepackage{lipsum}
\usepackage{color}
\usepackage{CJKutf8}
\usepackage{amsmath}
\usepackage{amssymb}
\usepackage{amsthm}
\usepackage{multirow}
\usepackage{titlesec}
\usepackage{enumerate}
\usepackage{mathrsfs}


% 定义各种环境
\newtheorem{theorem}{Theorem}[section]
\newtheorem{lemma}[theorem]{Lemma}
\newtheorem{corollary}[theorem]{Corollary}
\newtheorem{proposition}[theorem]{Proposition}
\newtheorem{example}[theorem]{Example}
\theoremstyle{definition}
\newtheorem{remark}[theorem]{Remark}
\newtheorem{definition}[theorem]{Definition}
\newtheorem{assumption}[theorem]{Assumption}

\def \supp {\mathop\mathrm{\,supp\,}}

% 设置 proof 的格式
% \renewcommand{\proofname}{\emph{Proof}}

% 基本信息
\title{收敛和弱收敛的一个等价刻画}

\begin{document}
\begin{CJK*}{UTF8}{gbsn}

\maketitle

\section{收敛的等价刻画}

\begin{theorem} \label{thm1}
    设 $ \{x_k\}_{k \in \mathbb{N}} $ 是 $ \mathbb{R}^n $ 中的有界数列.
    $ \{x_k\}_{k \in \mathbb{N}} $ 收敛到 $ x_0 $ 当且仅当 
    $ \{x_k\}_{k \in \mathbb{N}} $ 的任意收敛子列均收敛到 $ x_0 $.
\end{theorem}

\begin{proof}
    $ "\Rightarrow" $ 显然成立, 下证 $ "\Leftarrow" $. 
    设 $ \{x_k\}_{k \in \mathbb{N}} $ 的任意收敛子列均收敛到 $ x_0 $.
    反设 $ \{x_k\}_{k \in \mathbb{N}} $ 不收敛到 $ x_0 $, 
    则存在 $ \varepsilon_0 \in (0, \infty) $ 使得对 $ \forall N \in \mathbb{N} $, 存在 $ k > N $ 使得
    $$
        |x_k - x_0| > \varepsilon_0.
    $$
    故可取子列 $ \{x_{n_k}\}_{k \in \mathbb{N}} $ 使得对 $ \forall k \in \mathbb{N} $,
    $$
        |x_{n_k} - x_0| > \varepsilon_0.
    $$
    $ \{x_{n_k}\}_{k \in \mathbb{N}} $ 是有界数列, 故存在收敛子列, 不妨记为其本身.
    由 $ "\Leftarrow" $ 的假设知 $ \{x_{n_k}\}_{k \in \mathbb{N}} $ 收敛到 $ x_0 $, 矛盾.
    故 $ \{x_k\}_{k \in \mathbb{N}} $ 收敛到 $ x_0 $, Theorem \ref{thm1} 证毕.
\end{proof}

\begin{remark} \label{2}
    若 $ \mathbb{R}^n $ 中的有界数列不收敛, 则存在两个子列收敛到不同的极限.
\end{remark}

\section{弱收敛的等价刻画}

\begin{lemma} 
    设 $ E $ 是自反空间 $ \mathcal{X} $ 的子集. 则 $ E $ 有界当且仅当 $ E $ 弱列紧.
\end{lemma}

\begin{remark} \label{lem}
    自反空间中的有界数列必有弱收敛子列. 
\end{remark}

\begin{theorem} \label{thm2}
    设 $ \{x_k\}_{k \in \mathbb{N}} $ 是自反空间 $ \mathcal{X} $ 中的有界数列. 则
    $ \{x_k\}_{k \in \mathbb{N}} $ 弱收敛到 $ x_0 $ 当且仅当 
    $ \{x_k\}_{k \in \mathbb{N}} $ 的任意弱收敛子列均弱收敛到 $ x_0 $.
\end{theorem}

\begin{proof}
    $ "\Rightarrow" $ 显然成立, 下证 $ "\Leftarrow" $. 
    设 $ \{x_k\}_{k \in \mathbb{N}} $ 的任意弱收敛子列均弱收敛到 $ x_0 $.
    反设 $ \{x_k\}_{k \in \mathbb{N}} $ 不弱收敛到 $ x_0 $, 
    则存在 $ f \in \mathcal{X}^* $ 使得
    $$
        f(x_k) \nrightarrow f(x_0), \quad \text{as } k \to \infty.
    $$
    注意到 $ \{x_k\}_{k \in \mathbb{N}} $ 有界, 从而 $ \{f(x_k)\}_{k \in \mathbb{N}} $ 有界. 
    由此及 Remark \ref{2} 知, 存在收敛子列 $ \{f(x_{n_k})\}_{k \in \mathbb{N}} $ 
    和 $ \{f(x_{p_k})\}_{k \in \mathbb{N}} $ 使得
    \begin{equation} \label{equ}
        \lim_{k \to \infty} f(x_{n_k}) \neq \lim_{k \to \infty} f(x_{p_k}).
    \end{equation}
    又由 Remark \ref{lem} 知, $ \{x_{n_k}\}_{k \in \mathbb{N}} $ 和 $ \{x_{p_k}\}_{k \in \mathbb{N}} $
    均有弱收敛子列, 不妨记为其本身. 由 $ "\Leftarrow" $ 的假设知
    $$
        x_{n_k} \rightharpoonup x_0, \quad 
        x_{p_k} \rightharpoonup x_0, \quad \text{ as } k \to \infty.
    $$
    则由 \eqref{equ} 知,
    $$
        f(x_0) = \lim_{k \to \infty} f(x_{n_k}) 
               \neq \lim_{k \to \infty} f(x_{p_k}) 
               = f(x_0).
    $$
    矛盾. 从而 $ \{x_k\}_{k \in \mathbb{N}} $ 弱收敛到 $ x_0 $, Theorem \ref{thm2} 证毕.
\end{proof}

\begin{remark}
    若自反空间中的有界数列不弱收敛, 则存在两个子列弱收敛到不同的极限.
\end{remark}

\end{CJK*}
\end{document} 


