\documentclass[a4paper,11pt]{article}
\usepackage{graphicx}
\usepackage{subfigure}
\usepackage{geometry}
\usepackage{framed}
\usepackage{lipsum}
\usepackage{color}
\usepackage{amsmath}
\usepackage{amssymb}
\usepackage{amsthm}
\usepackage{multirow}
\usepackage{titlesec}
\usepackage{enumerate}
\usepackage{mathrsfs}
\usepackage{ctex}


% 定义各种环境
\newtheorem{theorem}{Theorem}[section]
\newtheorem{lemma}[theorem]{Lemma}
\newtheorem{corollary}[theorem]{Corollary}
\newtheorem{proposition}[theorem]{Proposition}
\newtheorem{example}[theorem]{Example}
\theoremstyle{definition}
\newtheorem{remark}[theorem]{Remark}
\newtheorem{definition}[theorem]{Definition}
\newtheorem{assumption}[theorem]{Assumption}

\def \supp {\mathop\mathrm{\,supp\,}}

% 设置 proof 的格式
% \renewcommand{\proofname}{\emph{Proof}}

% 基本信息
\title{单调函数的间断点集}

\begin{document}


\maketitle

\section{单调函数的间断点集}

众所周知, 单调函数的间断点集是可数集.

\section{双变量单调函数的间断点集}

首先需要定义什么是 $ \mathbb{R}^2 $ 上的单调函数.

\begin{definition}
    称 $ (x_1, y_1) \preceq (x_2, y_2) $, 若 $( x_1 \leq x_2 $) 且 $( x_1 \leq x_2 $).
    若函数 $ f :\ \mathbb{R}^2 \to \mathbb{R} $ 满足, 
    对任意 $ (x_1, y_1) \preceq (x_2, y_2) $, 有 $ f(x_1, y_1) \leq f(x_2, y_2) $, 则称 $ f $ 单调.
\end{definition}

\begin{theorem} \label{3}
    $ \mathbb{R}^2 $ 上单调函数的间断点集是零测度集.
\end{theorem}

\begin{proof}[证法一]
    设 $ f $ 是 $ \mathbb{R}^2 $ 上的单增函数. 对任意 $ n \in \mathbb{N} $
    将 $ [-n, n]^2 $ 均分, 计算达布上下和, 不难看出它们的差趋于 $ 0 $, 故 $ f $ 在 $ [-n, n]^2 $ 上黎曼可积,
    从而 $ f $ 在 $ [-n, n]^2 $ 上的间断点集 $ D_n(f) $ 是零测度集.
    因此 $ f $ 在 $ \mathbb{R}^2 $ 上的间断点集 
    $$
        D(f) = \bigcup_{n \in \mathbb{N}} D_n(f)
    $$
    也是零测度集. Theorem \ref{3} 证毕.
\end{proof}

另一种证明来自 [https://math.stackexchange.com/questions/
2338318/bivariate-function-monotone-in-each-variable-rightarrow-continuous-a-e].

\begin{proof}[证法二]
    设 $ f $ 是 $ \mathbb{R}^2 $ 上的单增函数且 $ D_f $ 为 $ f $ 在 $ \mathbb{R}^2 $ 上的间断点集.
    则
    \begin{align*}
        |D_f| &= \int_\mathbb{R} \int_\mathbb{R} \mathbf{1}_{D_f}(x, y) \, dx \, dy \\
              &= \frac{1}{2} \int_\mathbb{R} \left[ \int_\mathbb{R} \mathbf{1}_{D_f}(s - t, s + t) \, ds \right] \, dt
                & (x, y) := (s - t, s + t) \\
              &= \frac{1}{2} \int_\mathbb{R} |E_t| \, dt
                & E_t := \left\{ (s - t, s + t) \in D_f :\ s \in \mathbb{R} \right\}.
    \end{align*}
    下面计算 $ |E_t| $. 
    对任意固定 $ t \in \mathbb{R} $, 定义 $ g_t :\ \mathbb{R} \to \mathbb{R},\ s \mapsto f(s - t, s + t) $,
    由 $ f $ 是 $ \mathbb{R}^2 $ 上的单增函数知 $ g_t $ 在 $ \mathbb{R} $ 上单增.
    记 $ g_t $ 在 $ \mathbb{R} $ 上的间断点集为 $ D_{g_t} $, 则 $ D_{g_t} $ 是至多可数集. 断言 
    $$ 
        E_t \subset D_{g_t}.
    $$
    若断言成立, 则 $ |E_t| \leq |D_{g_t}| = 0 $. 从而
    $$
        |D_f| = \frac{1}{2} \int_\mathbb{R} |E_t| \, dt = 0.
    $$
    
    下证断言. 事实上, 对 $ \forall s_0 \in E_t $, 若 $ s_0 \notin D_{g_t} $,
    即 $ g_t $ 在 $ s_0 $ 点连续, 则对 $ \forall \varepsilon \in (0, \infty) $, 存在 $ \delta \in (0, \infty) $
    使得
    $$  
        g_t(s_0 + \delta) - \varepsilon \leq g_t(s_0) \leq g_t(s_0 - \delta) + \varepsilon,
    $$
    即
    \begin{equation}  \label{1}
        f(s_0 + \delta - t, s_0 + \delta + t) - \varepsilon
            \leq f(s_0 - t, s_0 + t) \leq f(s_0 - \delta - t, s_0 - \delta + t) + \varepsilon.
    \end{equation} 
    对 $ \forall (x, y) \in B((s_0 - t, s_0 + t), \delta) $, 有
    $$ 
        s_0 - \delta - t < x < s_0 + \delta - t
        \quad\text{和}\quad
        s_0 - \delta + t < y < s_0 + \delta + t,
    $$
    由此及 $ f $ 是 $ \mathbb{R}^2 $ 上的单增函数知
    $$
        f(s_0 - \delta - t, s_0 - \delta + t) \leq f(x, y) \leq f(s_0 + \delta - t, s_0 + \delta + t)
    $$
    结合 \eqref{1}, 我们有
    $$
        f(s_0 - t, s_0 + t) - \varepsilon \leq f(x, y) \leq f(s_0 - t, s_0 + t) + \varepsilon,
    $$
    即 $ f $ 在点 $ (s_0 - t, s_0 + t) $ 处连续, 从而 $ s_0 \notin E_t $, 矛盾. 因此 $ s_0 \in D_{g_t} $.
    由 $ s_0 $ 的任意性知, 断言成立. 
    Theorem \ref{3} 证毕.
\end{proof}

实际上, 上述证明还差一个重要事实.

\begin{theorem} \label{1}
    设 $ (\mathcal{X}, d_\mathcal{X}) $ 和 $ (\mathcal{Y}, d_\mathcal{Y}) $ 是度量空间,
    $ f :\ \mathcal{X} \to \mathcal{Y} $, $ A $ 是 $ f $ 在 $ \mathcal{X} $ 上的连续点集.
    则 $ A $ 是 $ G_\delta $ 集(可数个开集的交).
\end{theorem}

\begin{proof}
    对 $ \forall x \in \mathcal{X} $, 定义 $ f $ 在点 $ x $ 处的振动(oscillation)
    $$
        \omega_f(x) := \lim_{\delta \to 0^+} \sup_{y, z \in B_\mathcal{X}(x, \delta)} 
            d_\mathcal{Y} \left( f(x), f(y) \right) .
    $$
    $ f $ 在点 $ x $ 处连续当且仅当 $ \omega_f(x) = 0 $. 因此
    $$
        A = \bigcap_{n \in \mathbb{N}} A_n, 
        \quad\text{其中}\quad A_n := \left\{ x \in \mathcal{X} :\ \omega_f(x) < \frac{1}{n} \right\}.
    $$
    容易看出来 $ A_n $ 是开集, 故 $ A $ 是 $ G_\delta $ 集.
\end{proof}

\begin{remark}
    对于 Lebesgue 测度来说, $ G_\delta $ 集可测, 至此证明二才算圆满.
\end{remark}


\end{document} 









