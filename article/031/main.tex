\documentclass[a4paper,11pt]{article}
\usepackage{graphicx}
\usepackage{subfigure}
\usepackage{geometry}
\usepackage{framed}
\usepackage{lipsum}
\usepackage{color}
\usepackage{CJKutf8}
\usepackage{amsmath}
\usepackage{amssymb}
\usepackage{amsthm}
\usepackage{multirow}
\usepackage{titlesec}
\usepackage{enumerate}
\usepackage{mathrsfs}


% 定义各种环境
\newtheorem{theorem}{Theorem}[section]
\newtheorem{lemma}[theorem]{Lemma}
\newtheorem{corollary}[theorem]{Corollary}
\newtheorem{proposition}[theorem]{Proposition}
\newtheorem{example}[theorem]{Example}
\theoremstyle{definition}
\newtheorem{remark}[theorem]{Remark}
\newtheorem{definition}[theorem]{Definition}
\newtheorem{assumption}[theorem]{Assumption}

\def \supp {\mathop\mathrm{\,supp\,}}

% 设置 proof 的格式
% \renewcommand{\proofname}{\emph{Proof}}

% 基本信息
\title{笔记-J. Duoandikoetxea, Fourier Analysis}

\begin{document}
\begin{CJK*}{UTF8}{gbsn}

\maketitle

\begin{lemma} \label{lem1}
    设 $ N \in \mathbb{N} $, 开区间 $ \{I_k\}_{k = 1}^N \subset \mathbb{R} $ 且
    $$
        \bigcap_{k = 1}^N I_k \neq \emptyset.
    $$
    则存在 $ k_0 \in \{1, \ldots, N\} $ 使得
    $$
        \bigcup_{k = 1}^N I_k = I_{k_0},
    $$
    或存在 $ k_1, k_2 \in \{1, \ldots, N\} $ 使得
    $$
        \bigcup_{k = 1}^N I_k = I_{k_1} \cup I_{k_2}.
    $$
\end{lemma}

\begin{proof}
    若存在 $ k_0 \in \{1, \ldots, N\} $ 使得
    $$
        \bigcup_{k = 1}^N I_k = I_{k_0},
    $$
    Lemma \ref{lem1} 证毕. 
    否则由 $ \bigcap_{k = 1}^N I_k \neq \emptyset $ 知, 
    存在 $ a, b \in \mathbb{R} $, $ a < b $ 使得
    $$
        \bigcup_{\alpha \in A} I_k = (a, b).
    $$
    存在 $ k_1, k_2 \in \{1, \ldots, N\} $ 使得
    $$
        \inf I_{k_1} = a, \quad \text{and} \quad \sup I_{k_2} = b.
    $$
    由此及 $ I_{k_1} \cap I_{k_2} \neq \emptyset $ 知
    $$
        \bigcup_{k = 1}^N I_k = I_{k_1} \cup I_{k_2}.
    $$
\end{proof}

\begin{lemma}[Lemma 2.6]
    Let $ \{ I_\alpha \}_{\alpha \in A} $ be a collection of open intervals in $ \mathbb{R}^n $ 
    and let $ K $ be a compact set contained in their union. Then there exists a finite subcollection
    $ \{ I_j \} $ such that
    $$
        K \subset \bigcup_j I_j, \quad \text{and} \quad 
        \forall x \in \mathbb{R},\ \sum_j \mathrm{1}_{I_j} (x) \leq 2.
    $$
\end{lemma}

\begin{proof}
    由 $ K $ 是紧集知, 存在 $ \{I_k^{(0)}\}_{k = 1}^{N_0} \subset \{ I_\alpha \}_{\alpha \in A} $ 使得
    $$
        K \subset \bigcup_{k = 1}^N I_k^{(0)}.
    $$
    若存在 $ x_0 \in \mathbb{R} $ 使得 $ \sum_{k = 1}^{N_0} \mathrm{1}_{I_j^{(0)}} (x_0) \geq 3 $, 
    则由 Lemma \ref{lem1} 知,
    存在 $ \{I_k^{(1)}\}_{k = 1}^{N_1} \subset \{ I_\alpha \}_{\alpha \in A} $ 使得 $ N_1 \leq N_0 - 1 $ 且
    $$
        K \subset \bigcup_{k = 1}^{N_1} I_k^{(1)}.
    $$
    若存在 $ x_1 \in \mathbb{R} $ 使得 $ \sum_{k = 1}^{N_1} \mathrm{1}_{I_j^{(1)}} (x_1) \geq 3 $, 重复之前的过程.
    由于 $ \{N_k\} $ 严格减且大于 $ 0 $,
    故此过程经过有限次后会停止, 此时存在 $ \{I_k^{(s)}\}_{k = 1}^{N_s} \subset \{ I_\alpha \}_{\alpha \in A} $ 使得
    $$
        K \subset \bigcup_{k = 1}^{N_s} I_k^{(s)} \quad \text{and} \quad 
        \forall x \in \mathbb{R},\ \sum_{k = 1}^{N_s} \mathrm{1}_{I_k^{(s)}} (x) \leq 2.
    $$
\end{proof}



\end{CJK*}
\end{document} 


