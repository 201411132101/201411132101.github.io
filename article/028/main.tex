\documentclass[a4paper,11pt]{article}
\usepackage{graphicx}
\usepackage{subfigure}
\usepackage{geometry}
\usepackage{framed}
\usepackage{lipsum}
\usepackage{color}
\usepackage{CJKutf8}
\usepackage{amsmath}
\usepackage{amssymb}
\usepackage{amsthm}
\usepackage{multirow}
\usepackage{titlesec}
\usepackage{enumerate}
\usepackage{mathrsfs}


% 定义各种环境
\newtheorem{theorem}{Theorem}[section]
\newtheorem{lemma}[theorem]{Lemma}
\newtheorem{corollary}[theorem]{Corollary}
\newtheorem{proposition}[theorem]{Proposition}
\newtheorem{example}[theorem]{Example}
\theoremstyle{definition}
\newtheorem{remark}[theorem]{Remark}
\newtheorem{definition}[theorem]{Definition}
\newtheorem{assumption}[theorem]{Assumption}

\def \supp {\mathop\mathrm{\,supp\,}}

% 设置 proof 的格式
% \renewcommand{\proofname}{\emph{Proof}}

% 基本信息
\title{复分析总结}

\begin{document}
\begin{CJK*}{UTF8}{gbsn}

\maketitle

标注来源的时候, 英文表示 Stein 的复分析, 如: Theorem 2.2. 中文表示铁爷的书, 如: 命题1.2.2.

\section{解析函数基本性质}

以下均源自 \cite{ss10}.

\begin{definition}[p.7]
    开集 $ \Omega \subset \mathbb{C} $ 被称为连通的, 若不存在两个不交的非空开集 $ \Omega_1 $ 和 $ \Omega_2 $ 使得
    $$
        \Omega = \Omega_1 \cup \Omega_2.
    $$
    $ \mathbb{C} $ 中的连通开集被称为区域.
\end{definition}

\begin{definition}[p.93]
    粗略地讲, 
    称区域 $ \Omega $ 中拥有相同端点的两条曲线是同伦的, 若这其中一条能不离开区域 $ \Omega $ 连续的变换成另一条曲线.
\end{definition}

\begin{definition}[p.96]
    称区域 $ \Omega $ 是单连通的, 若 $ \Omega $ 中拥有相同端点的任意两条曲线是同伦的.
\end{definition}


\begin{definition}[p.20]
    粗略地讲, 
    一个光滑或者分段光滑曲线被称为是闭的, 若曲线首尾相连.    
    一个光滑或者分段光滑曲线被称为是简单的, 若曲线不自交.
\end{definition}

\begin{theorem}[Theorem 2.2, p.351]
    设 $ \Gamma $ 是一段简单的, 闭的, 分段光滑的曲线. 则 $ \Gamma^c $ 等于两个不交区域的并. 
    特别的, 这两个区域一个是有界单连通区域, 称它为 $ \Gamma $ 的内部;
    另一个是无界单连通区域, 称它为 $ \Gamma $ 的外部.
\end{theorem}

铁爷书中区域的定义太乱了, 我觉得按他的意思, 如果明确指出是区域, 那就是连通开集.

\begin{theorem}[命题1.2.2]
    设 $ f $ 定义在开集 $ \Omega \subset \mathbb{C} $ 上的复值函数, $ a \in \Omega $. 
    若 $ f $ 在点 $ a \in \Omega $ 处实可微, 
    则 $ f $ 在点 $ a \in \Omega $ 处复可微当且仅当 $ \frac{\partial f}{\partial \overline{z}}|_a = 0 $.
    若 $ f $ 在点 $ a \in \Omega $ 处复可微, 
    则 $ f $ 在点 $ a \in \Omega $ 处实可微且 $ \frac{\partial f}{\partial z}|_a = f'(a) $.
\end{theorem}

\begin{remark}
    $ \frac{\partial f}{\partial \overline{z}} $ 是复可微与否的关键.
    $ \frac{\partial f}{\partial z} $ 就是导数.
\end{remark}

\begin{theorem}[定理1.2.3]
    设 $ \omega $ 是有界开集且 $ f \in H(\omega) \cap C^1(\overline{\omega}) $, 则
    $$
        \int_{\partial \omega} f(z) dz = 0.
    $$
\end{theorem}

\begin{remark}
    上面这个是柯西定理很一般的形式, 比较抽象, 下面是一个更直观的版本.
    
    设 $ \Omega $ 改为单连通开集, 由柯西定理及定理2.1.8 $ H(\Omega) = H(\Omega) \cap C^1(\Omega) $ 知, 
    对任意简单的, 闭的, 分段光滑曲线 $ \Gamma \subset \Omega $,
    $$
        \int_{\partial \Gamma} f(z) dz = 0.
    $$
    大致是这个意思, 曲线的条件可能有偏差.
\end{remark}

\begin{theorem}[定理1.2.4] \label{1.2.4}
    设 $ \omega $ 是有界开集且 $ z \in \omega $. 若 $ f \in C^1(\overline{\omega}) $, 则
    $$
        f(z) = \frac{1}{2 \pi i} \int_{\partial \omega} \frac{f(\zeta)}{ \zeta - z} d\zeta
                + \frac{1}{\pi} \int_{\omega} \frac{\partial f}{\partial \overline{\zeta}} 
                    \frac{1}{ \zeta - z} d\lambda(\zeta).
    $$
    若 $ f \in H(\omega) \cap C^1(\overline{\omega}) $, 则
    $$
        f(z) = \frac{1}{2 \pi i} \int_{\partial \omega} \frac{f(\zeta)}{ \zeta - z} d\zeta.
    $$
\end{theorem}

\begin{remark}
    上面这个是柯西公式很一般的形式, 比较抽象, 下面是一个更直观的版本.
        
    设 $ \Omega $ 改为单连通开集, 由柯西公式及定理2.1.8 $ H(\Omega) = H(\Omega) \cap C^1(\Omega) $ 知, 
    对任意简单的, 闭的, 分段光滑曲线 $ \Gamma \subset \Omega $,
    $$
        f(z) = \frac{1}{2 \pi i} \int_{\Gamma} \frac{f(\zeta)}{ \zeta - z} d\zeta.
    $$
    大致是这个意思, 曲线的条件可能有偏差.
\end{remark}

\begin{theorem}[定理2.1.8] \label{2.1.8}
    设 $ \Omega \subset \mathbb{R}^n $ 是开集, 则
    $$
        H(\Omega) = H(\Omega) \cap C^1(\Omega).
    $$
\end{theorem}

\begin{remark}
    设 $ \Omega $ 是区域, 则 
    $$ 
        H(\Omega) = H(\Omega) \cap C^1(\Omega) \cap \{\text{无穷可微函数}\},
    $$
    即全纯函数无穷可微且一阶偏导连续.
\end{remark}

\begin{proof}
    由定理2.1.8 知, $ H(\Omega) = H(\Omega) \cap C^1(\Omega) $, 即全纯函数一阶偏导连续.
    由此及定理1.2.6 知,
    $$
        H(\Omega) = H(\Omega) \cap C^1(\Omega) \subset \{\text{无穷可微函数}\}.
    $$
    因此全纯函数无穷可微且一阶偏导连续.
\end{proof}

\begin{theorem}[定理 1.1.12]
    设 $ X $ 是连通空间且 $ E \subset X $ 既开又闭, 则 $ E \in \{\emptyset, X\} $.
\end{theorem}

\begin{remark}
    定理 1.1.12 是更一般的结论, 不过没有这个直观. 此结论可用于证明超级有用的最大模定理.
\end{remark}


\begin{theorem}[定理 1.2.10]
    设 $ \Omega $ 是有界开集且 $ f \in H(\Omega) \cap C(\overline{\Omega}) $, 
    则存在 $ z_0 \in \partial \Omega $ 使得
    $
        |f(z_0)| = \sup_{z \in \overline{\Omega}} |f(z)|.
    $
\end{theorem}

\begin{remark}
    记
    $
        M := \sup_{z \in \overline{\Omega}} |f(z)|.
    $
    则 $ f $ 有两种可能. (i) $ |f| \equiv M $; (ii) 对 $ \forall z \in \Omega $, $ |f(z)| < M $.
\end{remark}

\begin{theorem}[Hadamard, 铁爷 p.11]
    设 $ \{c_n\}_{n \in \mathbb{N}} \subset \mathbb{C} $, 幂级数
    $$
        \sum_{n=0}^\infty c_n (z - a)^n
    $$
    的收敛半径 $ R $ 满足
    $$
        \frac{1}{R} = \limsup_{n \to \infty} \sqrt[n]{|c_n|}.
    $$
    若 $ R \in (0, \infty] $, 则级数在 $ D(a, R) $ 中内闭一致收敛.
    若 $ R \in [0, \infty) $, 则级数在 $ \{z \in \mathbb{C} :\ |z - a| > R\} $ 中发散.
\end{theorem}

\begin{theorem}[定理 1.2.11 和定理 1.2.14]
    设 $ \Omega $ 是有界开集. 
    $ f \in H(\Omega) $ 当且仅当对 $ \forall a \in \Omega $, 存在一个圆盘 $ D(a, r) \subset \Omega $ 
    使得对 $ \forall z \in \Omega $,
    $$
        f(z) = \sum_{n=0}^\infty c_n (z - a)^n
    $$
    的幂级数.
\end{theorem}

\begin{remark}
    由这个定理也能看出来全纯函数无穷可微.
    注意, 导数是能算出来的, 直接对积分内部求导即可.
\end{remark}

\begin{theorem}[定理 1.2.16, Corollary 4.5]
    设 $ f \in H(\mathbb{C}) $ 且有界, 则 $ f $ 是常值函数.
\end{theorem}

\begin{remark}
    著名的 Liouville 定理, 必须牢记. 
    称 $ f $ 是整函数, 若 $ f \in H(\mathbb{C}) $.
\end{remark}

\begin{theorem}[定理 1.2.18]
    (零点的孤立性). 设 $ \Omega $ 是区域(连通开集), $ f \in H(\Omega) $ 且 $ f $ 不恒等于 $ 0 $.
    记 $ Z(f) := \{z \in \Omega :\ f(z) = 0\} $, 则 $ Z(f) $ 在 $ \Omega $ 中没有极限点.
    
    (零点处函数的分解). 设 $ a \in Z(f) $, 则存在唯一正整数 $ m \in \mathbb{N} $ 
    和非 $ 0 $ 函数 $ g \in H(\Omega) $, 使得对 $ \forall z \in \Omega $,
    $$
        f(z) = (z - a)^m g(z).
    $$
\end{theorem}

\begin{theorem}[Theorem 4.8]
    看待零点的孤立性的另一个角度.
    设 $ \Omega $ 是区域(连通开集), $ f \in H(\Omega) $.
    若存在点 $ z_0 \in \Omega $ 和序列 $ \{z_n\}_{n \in \mathbb{N}} \subset Z(f) \setminus \{z_0\} $ 使得
    $$
        z_0 = \lim_{n \to \infty} z_n,
    $$
    则 $ f \equiv 0 $.
\end{theorem}

\begin{remark}
    零点的孤立性, 非常有用. 由零点的孤立性, 容易看出 $ Z(f) $ 至多可数.
\end{remark}

\begin{theorem}[Theorem 1.2.20]
    设 $ a \in \mathbb{C} $, $ 0 \leq r < R < \infty $, 
    $$ \Omega := \{z \in \mathbb{C} :\ r < |z - a| < R \}. $$
    若 $ f \in H(\Omega) $, 则存在序列 $ \{c_n\}_{n \in \mathbb{Z}} \subset \mathbb{C} $ 使得, 
    对 $ \forall z \in \Omega $,
    $$
        f(z) = \sum_{n=-\infty}^\infty c_n (z - a).
    $$
    其中对 $ \forall n \in \mathbb{Z} $,
    $$
        c_n = \frac{1}{2 \pi i} \int_{\partial D(a, \rho)} \frac{f(\zeta)}{(\zeta - a)^{n+1}} d\zeta,
    $$
    其中 $ \rho \in (r, R) $, 由柯西公式, 上式右边的积分与 $ \rho $ 的选取无关.
\end{theorem}

\begin{remark}
    Laurent 展开, 非常有用. 没有坏点, 就用泰勒, 有坏点, 就只能用 Laurent 展开了.
\end{remark}

\begin{thebibliography}{99}

    \bibitem{ss10}  E. M. Stein and R. Shakarchi, Complex Analysis, Princeton University Press, 2010.
    
\end{thebibliography}

\end{CJK*}
\end{document} 


