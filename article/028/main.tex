\documentclass[a4paper,11pt]{article}
\usepackage{graphicx}
\usepackage{subfigure}
\usepackage{geometry}
\usepackage{framed}
\usepackage{lipsum}
\usepackage{color}
\usepackage{CJKutf8}
\usepackage{amsmath}
\usepackage{amssymb}
\usepackage{amsthm}
\usepackage{multirow}
\usepackage{titlesec}
\usepackage{enumerate}
\usepackage{mathrsfs}


% 定义各种环境
\newtheorem{theorem}{Theorem}[section]
\newtheorem{lemma}[theorem]{Lemma}
\newtheorem{corollary}[theorem]{Corollary}
\newtheorem{proposition}[theorem]{Proposition}
\newtheorem{example}[theorem]{Example}
\theoremstyle{definition}
\newtheorem{remark}[theorem]{Remark}
\newtheorem{definition}[theorem]{Definition}
\newtheorem{assumption}[theorem]{Assumption}

\def \supp {\mathop\mathrm{\,supp\,}}

% 设置 proof 的格式
% \renewcommand{\proofname}{\emph{Proof}}

% 基本信息
\title{Simple Connectivity and Jordan Curve Theorem}

\begin{document}
\begin{CJK*}{UTF8}{gbsn}

\maketitle

以下所有内容均源自 \cite{ss10}. 首先回顾一下复分析中的一些基本定义.

\begin{definition}[p.7]
    开集 $ \Omega \subset \mathbb{C} $ 被称为连通的, 若不存在两个不交的非空开集 $ \Omega_1 $ 和 $ \Omega_2 $ 使得
    $$
        \Omega = \Omega_1 \cup \Omega_2.
    $$
    $ \mathbb{C} $ 中的连通开集被称为区域.
\end{definition}

\begin{definition}[p.93]
    粗略地讲, 
    称区域 $ \Omega $ 中拥有相同端点的两条曲线是同伦的, 若这其中一条能不离开区域 $ \Omega $ 连续的变换成另一条曲线.
\end{definition}

\begin{definition}[p.96]
    称区域 $ \Omega $ 是单连通的, 若 $ \Omega $ 中拥有相同端点的任意两条曲线是同伦的.
\end{definition}


\begin{definition}[p.20]
    粗略地讲, 
    一个光滑或者分段光滑曲线被称为是闭的, 若曲线首尾相连.    
    一个光滑或者分段光滑曲线被称为是简单的, 若曲线不自交.
\end{definition}

\begin{theorem}[Theorem 2.2, p.351]
    设 $ \Gamma $ 是一段简单的, 闭的, 分段光滑的曲线. 则 $ \Gamma^c $ 等于两个不交区域的并. 
    特别的, 这两个区域一个是有界单连通区域, 称它为 $ \Gamma $ 的内部;
    另一个是无界单连通区域, 称它为 $ \Gamma $ 的外部.
\end{theorem}

\begin{thebibliography}{99}

    \bibitem{ss10}  E. M. Stein and R. Shakarchi, Complex Analysis, Princeton University Press, 2010.
    
\end{thebibliography}

\end{CJK*}
\end{document} 


