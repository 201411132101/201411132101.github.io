\documentclass[a4paper,11pt]{article}
\usepackage{graphicx}
\usepackage{subfigure}
\usepackage{geometry}
\usepackage{framed}
\usepackage{lipsum}
\usepackage{color}
\usepackage{amsmath}
\usepackage{amssymb}
\usepackage{amsthm}
\usepackage{multirow}
\usepackage{titlesec}
\usepackage{enumerate}
\usepackage{mathrsfs}
\usepackage{ctex}


% 定义各种环境
\newtheorem{theorem}{Theorem}[section]
\newtheorem{lemma}[theorem]{Lemma}
\newtheorem{corollary}[theorem]{Corollary}
\newtheorem{proposition}[theorem]{Proposition}
\newtheorem{example}[theorem]{Example}
\theoremstyle{definition}
\newtheorem{remark}[theorem]{Remark}
\newtheorem{definition}[theorem]{Definition}
\newtheorem{assumption}[theorem]{Assumption}

\def \supp {\mathop\mathrm{\,supp\,}}

% 设置 proof 的格式
% \renewcommand{\proofname}{\emph{Proof}}

% 基本信息
\title{等距同构的那些事}

\begin{document}


\maketitle

\section{度量空间的等距同构}

度量空间的等距同构没啥问题.

\begin{definition}
    设 $ (\mathscr{X}, d_{\mathscr{X}}),\ (\mathscr{Y}, d_{\mathscr{Y}}) $ 是度量空间.
    若 $ T :\ \mathscr{X} \to \mathscr{Y} $ 满足对 $ \forall x_1,\ x_2 \in \mathscr{X} $,
    $$
        d_{\mathscr{Y}} (T(x_1), T(x_2)) = d_{\mathscr{X}} (x_1, x_2),
    $$
    则称 $ T $ 是等距映射(isometry).
\end{definition}

\begin{definition}
    设 $ (\mathscr{X}, d_{\mathscr{X}}),\ (\mathscr{Y}, d_{\mathscr{Y}}) $ 是度量空间.
    若存在 $ \mathscr{X} $ 到 $ \mathscr{Y} $ 满的等距映射, 则称 $ \mathscr{X} $ 和 $ \mathscr{Y} $ 等距同构.
\end{definition}

\section{线性赋范空间的等距同构}

线性赋范空间的等距同构, 问题就来了. 首先, 线性赋范空间一定是度量空间, 
那么想要等距同构, 一定要满足度量空间等距同构的要求.
但是, 线性赋范空间有线性结构, 这个也要保持, 所以要求等距映射线性.

\begin{definition}
    设 $ (\mathscr{X}, \| \cdot \|_\mathscr{X}),\ (\mathscr{Y}, \| \cdot \|_\mathscr{Y}) $ 是线性赋范空间.
    若存在 $ \mathscr{X} $ 到 $ \mathscr{Y} $ 满的线性等距映射, 则称 $ \mathscr{X} $ 和 $ \mathscr{Y} $ 等距同构.
\end{definition}

实际上, 对于实线性赋范空间, 定义中的条件可以减弱成度量空间的条件.
这个是由 Mazur-Ulam Theorem 保证的. 而对于复线性赋范空间, 等距同构映射必须线性.

\begin{lemma}  \label{1}
    设 $ (\mathscr{X}, \| \cdot \|_\mathscr{X}),\ (\mathscr{Y}, \| \cdot \|_\mathscr{Y}) $ 是实线性赋范空间.
    若存在 $ \mathscr{X} $ 到 $ \mathscr{Y} $ 满的等距映射, 则存在 $ \mathscr{X} $ 到 $ \mathscr{Y} $ 满的线性等距映射.
\end{lemma}

在证明 Lemma \ref{1} 之前, 首先回顾一下 Mazur-Ulam Theorem 及其相关定义.

\begin{definition}
    设 $ (\mathscr{X}, \| \cdot \|_\mathscr{X}),\ (\mathscr{Y}, \| \cdot \|_\mathscr{Y}) $ 是实线性赋范空间.
    称 $ T :\ \mathscr{X} \to \mathscr{Y} $ 是仿射变换(affine), 若对 $ \forall a,\ b \in \mathcal{X} $
    和 $ \forall t \in \mathbb{R} $,
    $$
        T((1 - t) a + t b) = (1 - t) T(a) + t T(b).
    $$
\end{definition}

\begin{theorem}[Mazur-Ulam]
    设 $ (\mathscr{X}, \| \cdot \|_\mathscr{X}),\ (\mathscr{Y}, \| \cdot \|_\mathscr{Y}) $ 是实线性赋范空间.
    若 $ T :\ \mathscr{X} \to \mathscr{Y} $ 满的等距映射, 则 $ T $ 是仿射变换(affine).
\end{theorem}

现在可以开始证明 Lemma \ref{1} 了.

\begin{proof}[Proof of Lemma \ref{1}]
    若存在 $ \mathscr{X} $ 到 $ \mathscr{Y} $ 满的等距映射 $ T $, 由 Mazur-Ulam Theorem 知,
    $ T $ 是仿射变换. 令 $ S := T - T(\theta) $, 则对 $ \forall x_1,\ x_2 \in \mathcal{X} $,
    $$
        \| S(x_1) - S(x_2) \|_\mathscr{Y} 
            = \| T(x_1) - T(x_2) \|_\mathscr{Y} 
            = \| x_1 - x_2 \|_\mathscr{X},
    $$
    且对 $ \forall x \in \mathcal{X} $ 和 $ t \in \mathbb{R} $,
    $$
        S(t x) = T(t x) - T(\theta)
                 = T(t x + (1 - t) \theta) - T(\theta)
                 = t[T(x) - T(\theta)]
                 = t S(x),
    $$
    从而对 $ \forall x_1,\ x_2 \in \mathcal{X} $ 和 $ t \in \mathbb{R} $,
    \begin{align*}
        S(x_1 + x_2) &= 2 S \left( \frac{x_1 + x_2}{2} \right) 
                     = 2 \left[ T \left( \frac{x_1 + x_2}{2} \right) - T(\theta) \right] \\
                     &= [T(x_1) - T(\theta)] - [T(x_2) - T(\theta)]
                     = S(x_1) + S(x_2).
    \end{align*}
    综上, $ S $ 是 $ \mathscr{X} $ 到 $ \mathscr{Y} $ 满的线性等距映射. Lemma \ref{1} 证毕.
\end{proof}


\end{document} 









